% Slide 1: Propósito del Acuerdo
% --- DIAPOSITIVA 2 ---
\begin{frame}{Nuestra Estrategia: ¿Por qué empezar pequeños?}
    \begin{itemize}
        \item<1-> \textbf{Estrategia Adoptada:} ``Tamaño inferior a la demanda con ampliaciones posteriores''.
        \item<2-> \textbf{Mercado Potencial:} Identificamos una demanda de más de \textbf{270 colegios} con perfil científico en Bogotá.
        \item<3-> \textbf{Restricción Clave:} Nuestra capacidad real está ligada al \textbf{número de pasantes} de la Universidad Distrital por año.
        \item<4-> \textbf{Enfoque Inteligente:} Iniciar con un piloto enfocado nos permite:
        \begin{itemize}
            \item<5-> Validar nuestro modelo educativo.
            \item<6-> Asegurar casos de éxito iniciales.
            \item<7-> Crecer de manera sostenible y con bajo riesgo financiero.
        \end{itemize}
    \end{itemize}
\end{frame}



% --- DIAPOSITIVA 3 ---
\begin{frame}{El "Producto-Servicio"}
    \begin{block}{Un Proyecto de Grado Integral}
        Cada implementación requiere \textbf{600 horas} de trabajo por pasante.
    \end{block}
    
    \vfill
    \begin{alertblock}{Nuestro Valor Agregado}
        No entregamos un equipo, implementamos un programa educativo completo.
    \end{alertblock}
\end{frame}



% --- DIAPOSITIVA 4 ---
\begin{frame}{Cronograma de Implementación (10 meses por colegio)}
    \begin{block}{Semestre 1 - Desarrollo y Puesta en Marcha (320 horas)}
        \begin{itemize}
            \item \textbf{Mes 1-2 (120 horas):} Planificación y construcción del hardware.
            \item \textbf{Mes 3 (100 horas):} Configuración de software y preparación de guías didácticas.
            \item \textbf{Mes 4-5 (100 horas):} Instalación en el colegio y talleres de formación inicial.
        \end{itemize}
    \end{block}
    
    \begin{block}{Semestre 2 - Acompañamiento y Cierre (280 horas)}
        \begin{itemize}
            \item \textbf{Mes 6-7 (120 horas):} Asesoría a estudiantes en su primer proyecto de observación.
            \item \textbf{Mes 8 (80 horas):} Soporte técnico y evaluación de impacto.
            \item \textbf{Mes 9-10 (80 horas):} Sistematización de la experiencia e informe final de grado.
        \end{itemize}
    \end{block}
\end{frame}



% --- DIAPOSITIVA 5 ---
\begin{frame}{Capacidad de Impacto Directo por Implementación}
    \begin{alertblock}{Objetivo}
        Crear capacidad instalada y autonomía en cada institución.
    \end{alertblock}
    
    \begin{columns}[T] % Alinea las columnas en la parte superior
        \begin{column}{0.4\textwidth}
            \centering
            \begin{block}{Formación Docente}
                Capacitamos un núcleo de \textbf{2 a 3 docentes} líderes por colegio.
            \end{block}
        \end{column}
        \begin{column}{0.4\textwidth}
            \centering

            \begin{block}{Formación Estudiantil}
                Formamos un grupo pionero de \textbf{15 a 20 estudiantes}.
            \end{block}
        \end{column}
    \end{columns}
\end{frame}



% --- DIAPOSITIVA 6 ---
\begin{frame}{Proyección de Crecimiento a 5 Años}
    \frametitle{Un Crecimiento Controlado y Sostenible}
    
    \begin{table}
    \centering
    \resizebox{\textwidth}{!}{%
    \begin{tabular}{|l|c|c|c|c|c|}
        \hline
        \textbf{Capacidad} & \textbf{Año 1} & \textbf{Año 2} & \textbf{Año 3} & \textbf{Año 4} & \textbf{Año 5} \\ \hline
        \hline
        N° de Pasantes & 3 & 6 & 9 & 9 & 9 \\ \hline
        Implementación (Colegios) & 3 & 6 & 9 & 9 & 9 \\ \hline
        Formación (Docentes) & 6 - 9 & 12 - 18 & 18 - 27 & 18 - 27 & 18 - 27 \\ \hline
        Formación (Estudiantes) & 45 - 60 & 90 - 120 & 135 - 180 & 135 - 180 & 135 - 180 \\ \hline
    \end{tabular}%
    }
    \caption{Proyección de crecimiento hasta alcanzar la capacidad madura del proyecto.}
    \end{table}
    
    \vfill
    \begin{alertblock}{}
        Meta: Alcanzar una capacidad madura de 9 implementaciones anuales a partir del tercer año.
    \end{alertblock}
\end{frame}



% --- DIAPOSITIVA 7 ---
\begin{frame}{Conclusiones y Próximos Pasos}
    \begin{itemize}
        \item<1-> \textbf{Tamaño Inicial:} Iniciaremos con \textbf{3 implementaciones} en el Año 1 para validar el modelo y perfeccionar los procesos.
        \item<2-> \textbf{Enfoque Estratégico:} Priorizaremos colegios del clúster "Corredor Científico del Norte" para asegurar el éxito del piloto.
        \item<3-> \textbf{Visión a Futuro:} El plan garantiza un crecimiento escalonado hasta alcanzar una capacidad sostenible y de alto impacto.
        \item<4-> \textbf{Siguiente Paso:} Formalizar las alianzas estratégicas con los 3 colegios seleccionados.
    \end{itemize}
\end{frame}
