\section{Introducción}

\begin{frame}{Inversión del Proyecto}
    \begin{itemize}
        \item Los ingresos principales provendrán de la venta de 
            radiotelescopios y servicios asociados (capacitación y soporte a 
            escuelas privadas).
    \end{itemize}
\end{frame}

\section{Ingresos Proyectados}

\begin{frame}{Ingresos Proyectados (Año 1)}
    \begin{itemize}
        \item \textbf{Costo de construcción por unidad:} COP\$ 800.000 
            (incluyendo enseñanza para estudiantes).
        \item \textbf{Margen de beneficio bruto:} 30\%.
        \item \textbf{Precio de venta por unidad:} Aproximadamente COP\$ 1.142.857.
        \item \textbf{Ventas proyectadas (Año 1):} 3 radiotelescopios en 3 colegios.
        \item \textbf{Ingresos totales (Año 1):} 3 unidades $\times$ 
            COP\$ 1.142.857/unidad = COP\$ 3.428.571.
    \end{itemize}
\end{frame}

\section{Inversión Necesaria}

\begin{frame}{Inversión Necesaria}
    \begin{itemize}
        \item La inversión principal se destinará a la adquisición de 
            materiales y a cubrir los costos asociados con los estudiantes 
            pasantes.
        \item \textbf{Costo estimado por unidad:}
        \begin{itemize}
            \item Antenas y LNB:
            \begin{itemize}
                \item Servisistemas: COP\$ 460.000
                \item AZ Colombia Store: COP\$ 400.000 (precio de cinco antenas)
                \item Mercado Libre Colombia: COP\$ 200.000
            \end{itemize}
            \item Arduino R3:
            \begin{itemize}
                \item TD Electrónica: COP\$ 60.0000
                \item Mercado Libre Colombia: COP\$ 150.000
            \end{itemize}
            \item Potenciómetro:
            \begin{itemize}
                \item TD Electrónica: COP\$ 3.000
                \item Mercado Libre Colombia: COP\$ 4.000
            \end{itemize}
            \item Protoboard:
            \begin{itemize}
                \item TD Electrónica: COP\$ 20.000
                \item Mercado Libre Colombia: COP\$ 25.000
            \end{itemize}
        \end{itemize}
    \end{itemize}
\end{frame}

\begin{frame}
    \begin{itemize}
        \item Cables:
            \begin{itemize}
                \item TD Electrónica: COP\$ 15.000
                \item Vicartechz: COP\$ 10.000
                \item Mercado Libre Colombia: COP\$ 30.000
            \end{itemize}
        \end{itemize}
        \item \textbf{Costo promedio por unidad:} Incluye mano de obra 
            (COP\$ 300.000), totalizando COP\$ 800.000 para la construcción y 
            enseñanza.
        \item \textbf{Inversión total (Año 1):}
        \begin{itemize}
            \item 3 estudiantes pasantes, cada uno responsable de una 
                implementación.
            \item Costo de producción para las 3 unidades: 3 unidades 
                $\times$ COP\$ 800.000/unidad = COP\$ 2.400.000.
    \end{itemize}
\end{frame}

\section{Gestión de Crédito}

\begin{frame}{Gestión de Crédito}
    \begin{itemize}
        \item Se proponen dos modelos de financiación flexibles para facilitar 
            la adquisición.
        \item \textbf{Modelo 1:} Pago inicial completo por parte del colegio 
            antes del inicio del proyecto.
        \item \textbf{Modelo 2:}
        \begin{itemize}
            \item El proyecto gestiona un crédito a corto plazo con una entidad 
                bancaria (ej. Bancolombia, Davivienda, Banco de Bogotá).
            \item La institución educativa realiza el pago al finalizar la 
                implementación, momento en el cual el proyecto salda el préstamo.
        \end{itemize}
    \end{itemize}
\end{frame}
