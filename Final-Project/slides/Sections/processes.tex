\begin{frame}[fragile]
  \frametitle{Diagrama de Flujo del Proceso}

  \begin{figure}
    \centering
    \resizebox{0.4\textwidth}{!}{%
      \begin{tikzpicture}[
        node distance=0.8cm and 1.2cm, % y-dist and x-dist
        proceso/.style={
          rectangle, draw, fill=blue!10, text width=4.2cm, 
          text centered, minimum height=1.4cm, rounded corners=3pt
        },
        fase/.style={
          rectangle, draw, dashed, inner ysep=0.5cm,
          inner xsep=0.5cm, rounded corners=5pt
        },
        linea/.style={ -Stealth, draw, thick }
        ]

    % --- COLUMNA 1 (IZQUIERDA, BAJANDO) ---
        \node (A) [proceso] {Inicio: Convenio con Colegio};
        \node (B) [proceso, below=of A] {Planificación del Proyecto};
        \node (C) [proceso, below=of B] 
        {Adquisición de Componentes y Materiales};
        \node (D) [proceso, below=of C] 
        {Ensamblaje y Soldadura del Hardware};
        \node (E) [proceso, below=of D] {Pruebas y Calibración Inicial};
        \node (F) [proceso, below=of E] {Instalación de Software};
        \node (G) [proceso, below=of F] {Adaptación de Guías Didácticas};

    % --- COLUMNA 2 (DERECHA, SUBIENDO) ---
        \node (H) [proceso, right=of G, xshift=1.5cm] 
        {Logística e Instalación};
        \node (I) [proceso, above=of H]
        {Capacitación a Docentes y Estudiantes};
        \node (J) [proceso, above=of I] 
        {Asesoría en Proyectos de Observación};
        \node (K) [proceso, above=of J] {Soporte Técnico y Pedagógico};
        \node (L) [proceso, above=of K] {Evaluación de Impacto};
        \node (M) [proceso, above=of L] {Sistematización de Experiencia};
        \node (N) [proceso, above=of M] {Elaboración Informe Final};

    % --- FLECHAS ---
        \draw [linea] (A) -- (B); \draw [linea] (B) -- (C);
        \draw [linea] (C) -- (D); \draw [linea] (D) -- (E);
        \draw [linea] (E) -- (F); \draw [linea] (F) -- (G);
        \draw [linea] (G) -- (H);
        \draw [linea] (H) -- (I); \draw [linea] (I) -- (J);
        \draw [linea] (J) -- (K); \draw [linea] (K) -- (L);
        \draw [linea] (L) -- (M); \draw [linea] (M) -- (N);

    % --- CONTENEDORES DE FASE ---
        \node[fase, fit=(A)(E), 
        label={[anchor=north]north:Fase 1: Producción}] {};
        \node[fase, fit=(F)(G)(H)(I),
        label={[anchor=south]south:Fase 2: Implementación}] {};
        \node[fase, fit=(J)(L), 
        label={[anchor=south]south:Fase 3: Acompañamiento}] {};
        \node[fase, fit=(M)(N), 
        label={[anchor=north]north:Fase 4: Cierre}] {};

      \end{tikzpicture}
    } % fin de resizebox
  \end{figure}
\end{frame}

% --- DIAPOSITIVA DE INSUMOS ---
\begin{frame}
  \frametitle{Recursos: Insumos y Materiales}
  \framesubtitle{Costo por unidad: \textbf{COP~\num{750000}}}

  \begin{columns}[T] % Columnas alineadas por arriba
    \begin{column}{0.5\textwidth}
      \begin{block}{Componentes Electrónicos}
        \begin{itemize}
          \item Antena parabólica
          \item LNB (Low-Noise Block)
          \item RTL-SDR Dongle
          \item Raspberry Pi 4
          \item MicroSD y Cables
          \item Fuente de alimentación
        \end{itemize}
      \end{block}
      \begin{block}{Material Pedagógico}
        \begin{itemize}
          \item Guías didácticas
          \item Manuales de usuario
          \item Material para talleres
        \end{itemize}
      \end{block}
    \end{column}
    \begin{column}{0.5\textwidth}
      \begin{block}{Materiales Estructurales}
        \begin{itemize}
          \item Trípode o base
          \item Caja protectora
          \item Tornillería y herrajes
        \end{itemize}
      \end{block}
      \begin{block}{Software}
        \begin{itemize}
          \item S.O. para Raspberry Pi
          \item Software de radioastronomía
        \end{itemize}
      \end{block}
    \end{column}
  \end{columns}
\end{frame}

% --- DIAPOSITIVA DE RECURSOS HUMANOS ---
\begin{frame}
  \frametitle{Recursos: Talento Humano}

  \resizebox{\textwidth}{!}{%
    \begin{tabular}{lll}
      \toprule
      \textbf{Rol} & \textbf{Perfil Requerido} & \textbf{Responsabilidades Clave} \\
      \midrule
      \textbf{Pasante} & Estudiante de Ing./Física & 
      Liderar todo el ciclo del proyecto. \\
                       & (600h / 10 meses) &
                       (Producción, instalación, soporte). \\
                       \addlinespace
      \textbf{Tutor Académico} & Docente U.D. (STEM) & 
      Supervisar rigor técnico y académico. \\
                               & (2-4 h/semana) &
                               Facilitar recursos y evaluar. \\
                               \addlinespace
      \textbf{Coordinador} & Profesional en gestión & 
      Gestionar relación con colegios. \\
                           & (4-6 h/semana) &
                           Asegurar calidad y logística. \\
                           \addlinespace
      \textbf{Docente Contacto} & Docente del colegio & 
      Punto de contacto institucional. \\
                                & (Dedicación variable) &
                                Coordinar y multiplicar conocimiento. \\
                                \bottomrule
    \end{tabular}%
  }
\end{frame}

% --- DIAPOSITIVA DE MAQUINARIA ---
\begin{frame}
  \frametitle{Recursos: Maquinaria y Equipo}

  \begin{tabularx}{\textwidth}{lX}
    \toprule
    \textbf{Tipo de Equipo} & \textbf{Descripción y Uso Principal} \\
    \midrule
    \textbf{Herramientas Taller} & 
    Cautín, multímetro, alicates. Para el ensamblaje electrónico en el 
    laboratorio de la universidad. \\
    \addlinespace
    \textbf{Equipo de Cómputo} & 
    Computador portátil para programación, análisis de datos y elaboración 
    de informes. \\
    \addlinespace
    \textbf{Equipo de Pruebas} & 
    Osciloscopio y generador de señales (opcional) para calibración y 
    verificación del receptor. \\
    \addlinespace
    \textbf{Transporte} & 
    Vehículo para el traslado del equipo y del personal al colegio para la 
    instalación y seguimiento. \\
    \bottomrule
  \end{tabularx}
\end{frame}
