\begin{frame}{Estudio de mercado}
	\begin{itemize}
    \item Radiotelescopios de bajo costo.
    \item Posible financiación de instituciones privadas.
    \item Referencia: plan de la Secretaría de Educación con el IDARTES 
		(32 colegios en 12 localidades).
    \item Nuestra propuesta inicial: 10 instituciones educativas.
	\end{itemize}
\end{frame}

\begin{frame}{Estudio técnico}
	\begin{itemize}
    \item Antena parabólica
    \item Arduino R3
    \item Computador con software adecuado
    \item Buscador de satélites
    \item Circuito amplificador para aumentar la señal
	\end{itemize}
\end{frame}

\begin{frame}{Diagnóstico y selección de instituciones}
	\begin{itemize}
    \item Identificación de instituciones con potencial para albergar un 
		radiotelescopio.
    \item Factores evaluados: ubicación, infraestructura y disposición de la 
		comunidad educativa.
    \item Uso de la base de datos del gobierno.
    \item Consideración de experiencias previas en astronomía.
    \item Ejemplo: instituciones en el evento Astrofest (Universidad Sergio 
		Arboleda):
	\end{itemize}
\end{frame}

\begin{frame}{Diseño y planificación}
	\begin{itemize}
    \item Desarrollo de un plan de implementación detallado.
    \item Inclusión de cronogramas, responsables y actividades específicas.
    \item Definición de especificaciones de cada radiotelescopio.
    \item Integración con el currículo escolar.
    \item Diseño de protocolos de operación y seguridad.
	\end{itemize}
\end{frame}

\begin{frame}{Instalación y puesta en marcha}
	\begin{itemize}
    \item Instalación de radiotelescopios en colegios seleccionados.
    \item Ubicación estratégica para minimizar interferencias y maximizar 
		recepción.
    \item Realización de pruebas de funcionamiento.
    \item Calibración para garantizar operatividad óptima.
	\end{itemize}
\end{frame}

\begin{frame}{Capacitación docente y estudiantil}
	\begin{itemize}
    \item Capacitación docente en el uso del radiotelescopio.
    \item Formación técnica para estudiantes.
    \item Participación estudiantil en la construcción de los equipos.
    \item Fomento del aprendizaje activo y práctico.
	\end{itemize}
\end{frame}

\begin{frame}{Integración curricular}
	\begin{itemize}
    \item Elaboración de guías didácticas y materiales educativos.
    \item Facilitación de la incorporación del radiotelescopio en clases de física, matemáticas y tecnología.
    \item Desarrollo de metodologías activas.
    \item Aplicación de conceptos teóricos en la práctica por parte de los estudiantes.
	\end{itemize}
\end{frame}

\begin{frame}{Monitoreo y evaluación}
	\begin{itemize}
    \item Seguimiento al uso y estado de los equipos.
    \item Evaluación del impacto educativo.
    \item Retroalimentación para mejoras futuras.
	\end{itemize}
\end{frame}
