\section{Gastos y costos del proyecto} 

\begin{itemize}
    \item \textbf{Costos Directos (Costo de Bienes Vendidos - CBV):}
    \begin{itemize}
        \item El costo por unidad es de COP\$ 800.000.
        \item Los costos directos totales para el Año 1 ascienden a: 
					3 unidades $\times$ COP\$ 800.000/unidad = COP\$ 2.400.000.
    \end{itemize}
    \item \textbf{Costos y gastos Operativos:}
    \begin{itemize}
        \item \textbf{Estipendio del Pasante:} Los estudiantes colaborarán como 
					pasantes y su contribución formará parte de su trabajo de grado. Por 
					lo tanto, no se ha contemplado un costo salarial directo 
					significativo para el proyecto en este rubro, aprovechando el 
					contexto académico y el enfoque de bajo costo.
        \item \textbf{Capacitación docente:} Ya está incluido en el costo 
					unitario de COP\$ 800.000.
        \item \textbf{Gestión/Supervisión del proyecto:} La supervisión de los 
					estudiantes es asumida por tutores docentes adscritos al semillero de 
					Investigación en Energías Renovables de la Universidad.
        \item \textbf{Marketing y divulgación:} Nuestros planes incluyen 
					contactar a la Red Colombiana de Astronomía (RECA) que cuenta con un 
					programa denominado "La astronomía va a tu colegio", y a los colegios 
					seleccionados directamente. Los costos asociados son mínimos, 
					principalmente de comunicación y desplazamientos para reuniones.
        \item \textbf{Logística y viajes:} Para el primer año, hemos 
					seleccionado un clúster de colegios geográficamente cercanos en el 
					Eje Norte de Bogotá (Usaquén y Suba) para optimizar la logística y 
					reducir los tiempos de viaje. Asumimos 3 visitas por colegio 
					implementado (una para instalación y dos para seguimiento/soporte).
        \begin{itemize}
            \item Los costos de viaje para el Año 1 se estiman en: 
							3 colegios $\times$ 3 visitas/colegio $\times$ 
							COP\$ 50.000/visita = COP\$ 450.000.
        \end{itemize}
        \item \textbf{Gastos administrativos:} Hemos estimado los gastos 
					administrativos en un 5\% de los costos directos.
        \begin{itemize}
            \item Los gastos administrativos para el Año 1 son: 
							0.05 $\times$ COP\$ 2.400.000 = COP\$ 120.000.
        \end{itemize}
        \item \textbf{Total de Gastos Operativos (Año 1):} 
					COP\$ 450.000 + COP\$ 120.000 = COP\$ 570.000.
    \end{itemize}
\end{itemize}

\subsection*{Utilidad Bruta}

\begin{itemize}
    \item \textbf{Utilidad Bruta (Año 1):} Calculamos la utilidad bruta para el
			Año 1 como: 
			COP\$ 3.428.571 (Ingresos) - COP\$ 2.400.000 (Costos Directos) = COP\$ 1.028.571.
\end{itemize}

\subsection*{Utilidad Neta}

\begin{itemize}
    \item \textbf{Utilidad Neta (Año 1):} La utilidad neta para el Año 1 es: 
			COP\$ 1.028.571 (Utilidad Bruta) - COP\$ 570.000 (Gastos Operativos) = 
			COP\$ 458.571.
\end{itemize}

\section*{Estudio Financiero Proyectado (Años 2-5)}

Tomando la Proyección del Crecimiento Escalonado del Proyecto calcualda 
anteriormente:

\subsection*{Cálculos para el Año 2}
\begin{itemize}
    \item \textbf{Ingresos:} 6 unidades $\times$ COP\$ 1.142.857/unidad = 
			COP\$ 6.857.142.
    \item \textbf{Inversión (Costo de producción):} 
			6 unidades $\times$ COP\$ 800.000/unidad = COP\$ 4.800.000.
    \item \textbf{Costos Directos (CBV):} COP\$ 4.800.000.
    \item \textbf{Gastos Operativos:}
    \begin{itemize}
        \item Costos de viaje: 6 colegios $\times$ 3 visitas/colegio $\times$ 
					COP\$ 50.000/visita = COP\$ 900.000.
        \item Gastos administrativos: 
					0.05 $\times$ COP\$ 4.800.000 = COP\$ 240.000.
        \item Total: COP\$ 1.140.000.
    \end{itemize}
    \item \textbf{Utilidad Bruta:} COP\$ 6.857.142 - COP\$ 4.800.000 = COP\$ 2.057.142.
    \item \textbf{Utilidad Neta:} COP\$ 2.057.142 - COP\$ 1.140.000 = COP\$ 917.142.
\end{itemize}

\subsection*{Cálculos para el Año 3}
\begin{itemize}
    \item \textbf{Ingresos:} 9 unidades $\times$ COP\$ 1.142.857/unidad = 
			COP\$ 10.285.713.
    \item \textbf{Inversión (Costo de producción):} 
			9 unidades $\times$ COP\$ 800.000/unidad = COP\$ 7.200.000.
    \item \textbf{Costos Directos (CBV):} COP\$ 7.200.000.
    \item \textbf{Gastos Operativos:}
    \begin{itemize}
        \item Costos de viaje: 
					9 colegios $\times$ 3 visitas/colegio $\times$ COP\$ 50.000/visita = 
					COP\$ 1.350.000.
        \item Gastos administrativos: 0.05 $\times$ COP\$ 7.200.000 = 
					COP\$ 360.000.
        \item Total: COP\$ 1.710.000.
    \end{itemize}
    \item \textbf{Utilidad Bruta:} COP\$ 10.285.713 - COP\$ 7.200.000 = COP\$ 3.085.713.
    \item \textbf{Utilidad Neta:} COP\$ 3.085.713 - COP\$ 1.710.000 = COP\$ 1.375.713.
\end{itemize}

\subsection*{Cálculos para el Año 4}
\begin{itemize}
    \item \textbf{Ingresos:} 
			9 unidades $\times$ COP\$ 1.142.857/unidad = COP\$ 10.285.713.
    \item \textbf{Inversión (Costo de producción):} 
			9 unidades $\times$ COP\$ 800.000/unidad = COP\$ 7.200.000.
    \item \textbf{Costos Directos (CBV):} COP\$ 7.200.000.
    \item \textbf{Gastos Operativos:}
    \begin{itemize}
        \item Costos de viaje: 
					9 colegios $\times$ 3 visitas/colegio $\times$ COP\$ 50.000/visita = 
					COP\$ 1.350.000.
        \item Gastos administrativos: 
					0.05 $\times$ COP\$ 7.200.000 = COP\$ 360.000.
        \item Total: COP\$ 1.710.000.
    \end{itemize}
    \item \textbf{Utilidad Bruta:} COP\$ 10.285.713 - COP\$ 7.200.000 = COP\$ 3.085.713.
    \item \textbf{Utilidad Neta:} COP\$ 3.085.713 - COP\$ 1.710.000 = COP\$ 1.375.713.
\end{itemize}

\subsection*{Cálculos para el Año 5}
\begin{itemize}
    \item \textbf{Ingresos:} 
			9 unidades $\times$ COP\$ 1.142.857/unidad = COP\$ 10.285.713.
    \item \textbf{Inversión (Costo de producción):} 
			9 unidades $\times$ COP\$ 800.000/unidad = COP\$ 7.200.000.
    \item \textbf{Costos Directos (CBV):} COP\$ 7.200.000.
    \item \textbf{Gastos Operativos:}
    \begin{itemize}
        \item Costos de viaje: 
					9 colegios $\times$ 3 visitas/colegio $\times$ COP\$ 50.000/visita = 
					COP\$ 1.350.000.
        \item Gastos administrativos: 0.05 $\times$ COP\$ 7.200.000 = 
					COP\$ 360.000.
        \item Total: COP\$ 1.710.000.
    \end{itemize}
    \item \textbf{Utilidad Bruta:} COP\$ 10.285.713 - COP\$ 7.200.000 = COP\$ 3.085.713.
    \item \textbf{Utilidad Neta:} COP\$ 3.085.713 - COP\$ 1.710.000 = COP\$ 1.375.713.
\end{itemize}

\newpage

\section*{Resumen Proyectado de Utilidad Neta:}

\begin{table}[h!]
    \centering
    \small 
    \begin{tabularx}{\textwidth}{lXXXXX} 
        \toprule
        \textbf{Indicador} & \textbf{Año 1 (COP\$)} & \textbf{Año 2 (COP\$)} & \textbf{Año 3 (COP\$)} & \textbf{Año 4 (COP\$)} & \textbf{Año 5 (COP\$)} \\
        \midrule
        \textbf{Ingresos} &  3.428.571 & 6.857.142 & 10.285.713 & 10.285.713 & 10.285.713 \\
        \textbf{Costos Directos} & 2.400.000 & 4.800.000 & 7.200.000 & 7.200.000 & 7.200.000 \\
        \textbf{Utilidad Bruta} & 1.028.571 & 2.057.142 & 3.085.713 & 3.085.713 & 3.085.713 \\
        \textbf{Gastos Operativos} & 570.000 & 1.140.000 & 1.710.000 & 1.710.000 & 1.710.000 \\
        \textbf{Utilidad Neta} & 458.571 & 917.142 & 1.375.713 & 1.375.713 & 1.375.713 \\
        \bottomrule
    \end{tabularx}
    \caption{Resumen Proyectado de Utilidad Neta (Años 1-5).}
\end{table}

Este análisis muestra un crecimiento constante en la utilidad neta a medida que 
el proyecto escala su capacidad de implementación hasta alcanzar su madurez en 
el Año 3. La estrategia de crecimiento gradual y autofinanciado que hemos 
planteado permite una operación sostenible, generando los recursos necesarios 
para posibles expansiones futuras, incluyendo la fase de implementación en 
colegios públicos.
