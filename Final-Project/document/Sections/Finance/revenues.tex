Los ingresos principales provendrán de la venta de los radiotelescopios y los 
servicios asociados de capacitación y soporte a escuelas privadas.

\begin{itemize}
    \item \textbf{Precio de venta por unidad:} Hemos estimado el costo de 
			construcción de cada radiotelescopio, incluyendo la enseñanza para los 
			estudiantes, en COP\$ 800.000. Aplicando un margen de beneficio bruto del 
			30\%, el precio de venta por unidad se ha fijado en aproximadamente 
			COP\$ 1.142.857.
    \item \textbf{Ventas proyectadas (Año 1):} Para el primer año, planeamos 
			implementar 3 radiotelescopios en 3 colegios.
    \item \textbf{Ingresos totales (Año 1):} Calculamos que los ingresos 
			totales para el Año 1 serán de 3 unidades 
			$\times$ COP\$ 1.142.857/unidad = COP\$ 3.428.571.
\end{itemize}

\subsection*{Inversión Necesaria}

La inversión principal se destinará a la adquisición de materiales para los 
radiotelescopios y a cubrir los costos asociados con los estudiantes pasantes.

\begin{itemize}
    \item \textbf{Costo estimado por unidad:}
    Antenas y LNB:                                        
		\begin{itemize}
        \item \textbf{Servisistemas:} COP\$ 460.000
        \item \textbf{AZ Colombia Store:} COP\$ 400.000 (precio de cinco antenas)     
        \item \textbf{{Mercado Libre Colombia:}} COP \$ 200.000
    \end{itemize}
    Arduino R3:           
    \begin{itemize}
        \item \textbf{TD Electrónica:} COP\$ 60.0000
        \item \textbf{Mercado Libre Colombia:} COP\$ 150.000
    \end{itemize}                                                      
		Potenciometro:                                                     
		\begin{itemize}
        \item \textbf{TD Electrónica:} COP\$ 3.000
        \item \textbf{Mercado Libre Colombia:} COP\$ 4.000
    \end{itemize}                                                      
		Protoboard:
    \begin{itemize}
        \item \textbf{TD Electrónica:} COP\$ 20.000
        \item \textbf{Mercado Libre Colombia:} COP\$ 25.000
    \end{itemize}       
    Cables:                                                           
    \begin{itemize}
        \item \textbf{TD Electrónica:} COP\$ 15.000
        \item \textbf{Vicartechz:} COP\$ 10.000
        \item \textbf{Mercado Libre Colombia:} COP\$ 30.000
    \end{itemize}

		Se tiene previsto un costo de COP\$ 300.000 para la mano de obra, por lo 
		tanto, el costo promedio para la construcción de los radiotelescopios y la 
		enseñanza para los estudiantes es de COP\$ 800.000 por unidad.
   
    \item \textbf{Inversión total (Año 1):} En el primer año, el proyecto 
			contará con 3 estudiantes pasantes, y cada uno será responsable de una 
			implementación.
    \begin{itemize}
        \item El costo de producción para las 3 unidades asciende a: 
					3 unidades $\times$ COP\$ 800.000/unidad = COP\$ 2.400.000.
    \end{itemize}
\end{itemize}

\subsection*{Gestión de Crédito}
Para facilitar la adquisición por parte de las instituciones, proponemos dos 
modelos de financiación flexibles. El primero es un pago inicial completo por 
parte del colegio antes del inicio del proyecto, lo que asegura los fondos de 
manera directa. El segundo implica que el proyecto gestione un crédito a corto 
plazo con una entidad bancaria, para lo cual exploraremos líneas de crédito 
empresarial con bancos reconocidos como Bancolombia, Davivienda o Banco de 
Bogotá. La institución educativa se comprometería a realizar el pago al 
finalizar la implementación, momento en el cual el proyecto saldaría el 
préstamo.
