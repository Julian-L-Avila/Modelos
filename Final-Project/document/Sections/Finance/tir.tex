\section{Indicadores de Viabilidad Financiera}

Para evaluar la atractividad de la inversión en el proyecto, se calcula la
Tasa Interna de Retorno (TIR).

\subsection{Tasa Interna de Retorno (TIR)}

La TIR es la tasa de descuento que hace que el Valor Actual Neto (VAN) de los
flujos de caja de un proyecto sea igual a cero. En términos simples,
representa la rentabilidad porcentual promedio anual que genera el proyecto.

Para su cálculo, se utiliza el siguiente flujo de caja, donde el Año 0
representa la inversión inicial (costos directos del primer año) y los años
siguientes representan la Utilidad Operacional (UAII) generada:

\begin{itemize}
    \item \textbf{Flujo Año 0:} \num{-2400000} (Inversión)
    \item \textbf{Flujo Año 1:} \num{458571}
    \item \textbf{Flujo Año 2:} \num{917142}
    \item \textbf{Flujo Año 3:} \num{1375713}
    \item \textbf{Flujo Año 4:} \num{1375713}
    \item \textbf{Flujo Año 5:} \num{1375713}
\end{itemize}

Con base en estos flujos, la TIR del proyecto es de aproximadamente:
\[ \textbf{TIR} \approx \textbf{29.1\%} \]

Una TIR del 29.1\% es considerablemente alta y superior a la mayoría de las
tasas de oportunidad o costos de capital del mercado (usualmente entre 15\% y
20\%), lo que indica que el proyecto no solo es financieramente viable, sino
también una inversión muy atractiva.

Adicionalmente, es crucial destacar que el flujo de caja acumulado del
proyecto se torna positivo durante el transcurso del tercer año. Esto
significa que la inversión inicial de \mbox{COP \num{-2400000}} es recuperada
en su totalidad en este punto. Este hito financiero coincide con el último
escalamiento operativo a 9 colegios, validando que este tamaño no solo es
sostenible, sino que consolida la rentabilidad y viabilidad a largo plazo
del modelo de negocio.
