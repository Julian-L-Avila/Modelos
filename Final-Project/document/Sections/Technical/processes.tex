\section{Proceso de Producción}

En esta sección se detalla el conjunto de actividades, recursos y
procedimientos necesarios para llevar a cabo la producción e implementación
del producto-servicio, que consiste en un radiotelescopio educativo y el
acompañamiento pedagógico asociado. El proceso se ha diseñado para ser
eficiente, escalable y garantizar un alto estándar de calidad en cada
entrega.

\subsection{Diagrama de Flujo del Proceso}

El proceso completo, desde la planificación inicial hasta el cierre del
proyecto con cada colegio, se puede representar mediante el siguiente
diagrama de flujo. Este abarca tanto las fases de producción del hardware
como las de implementación y soporte pedagógico, reflejando la naturaleza
integral del servicio.

% --- INICIO DEL DIAGRAMA TIKZ (DISEÑO EN U) ---
\begin{figure}[H]
\centering
\begin{tikzpicture}[
    node distance=0.8cm and 1.5cm, % y-dist and x-dist
    proceso/.style={
        rectangle,
        draw,
        fill=blue!10,
        text width=4.5cm,
        text centered,
        minimum height=1.5cm,
        rounded corners=3pt
    },
    fase/.style={
        rectangle,
        draw,
        dashed,
        inner ysep=0.5cm,
        inner xsep=0.5cm,
        rounded corners=5pt
    },
    linea/.style={ -Stealth, draw, thick }
]

% --- COLUMNA 1 (IZQUIERDA, BAJANDO) ---
\node (A) [proceso] {Inicio: Convenio con Colegio};
\node (B) [proceso, below=of A] {Planificación del Proyecto};
\node (C) [proceso, below=of B]
    {Adquisición de Componentes Electrónicos y Materiales};
\node (D) [proceso, below=of C]
    {Ensamblaje y Soldadura del Hardware del Radiotelescopio};
\node (E) [proceso, below=of D] {Pruebas Internas y Calibración Inicial};
\node (F) [proceso, below=of E]
    {Instalación y Configuración del Software};
\node (G) [proceso, below=of F]
    {Desarrollo y Adaptación de Guías Didácticas};
\node (H) [proceso, below=of G]
    {Logística: Transporte e Instalación en Colegio};

% --- COLUMNA 2 (DERECHA, SUBIENDO) ---
\node (I) [proceso, below=of H]
    {Capacitación Inicial a Docentes y Estudiantes};
\node (J) [proceso, right=of I, xshift=2.5cm]
    {Asesoría para Proyectos de Observación Estudiantil};
\node (K) [proceso, above=of J] {Soporte Técnico y Pedagógico Remoto};
\node (L) [proceso, above=of K] {Evaluación de Impacto y Apropiación};
\node (M) [proceso, above=of L]
    {Sistematización de la Experiencia};
\node (N) [proceso, above=of M] {Elaboración de Informe Final de Pasantía};
\node (O) [proceso, above=of N] {Fin del Ciclo};

% --- FLECHAS DE CONEXIÓN ---
\draw [linea] (A) -- (B); \draw [linea] (B) -- (C);
\draw [linea] (C) -- (D); \draw [linea] (D) -- (E);
\draw [linea] (E) -- (F); \draw [linea] (F) -- (G);
\draw [linea] (G) -- (H);
\draw [linea] (H) -- (I); % Conexión entre columnas
\draw [linea] (I) -- (J); \draw [linea] (J) -- (K);
\draw [linea] (K) -- (L); \draw [linea] (L) -- (M);
\draw [linea] (M) -- (N); \draw [linea] (N) -- (O);

% --- CONTENEDORES DE FASE ---
\node[fase, fit=(A)(E),
    label={[anchor=north]north:Fase 1: Producción (Mes 1-2)}] {};
\node[fase, fit=(F)(H)(I),
    label={[anchor=north]south:Fase 2: Implementación (Mes 3-5)}] {};
\node[fase, fit=(J)(L),
    label={[anchor=north]south:Fase 3: Acompañamiento (Mes 6-8)}] {};
\node[fase, fit=(M)(O),
    label={[anchor=north]north:Fase 4: Cierre Académico (Mes 9-10)}] {};

\end{tikzpicture}
\caption{Flujo del proceso integral con diseño en U.}
\label{fig:flujo_proceso_u}
\end{figure}
% --- FIN DEL DIAGRAMA TIKZ ---

\subsection{Tabla de Insumos y Materiales}
A continuación, se detallan los insumos necesarios para la producción de
\textbf{una (1) unidad} de radiotelescopio y su correspondiente material
pedagógico.

\begin{longtable}{p{4cm} p{4.5cm} r r}
\caption{Tabla de Insumos y Materiales por Unidad.} \label{tab:insumos}\\
\toprule
\textbf{Categoría} & \textbf{Insumo/Material} & \textbf{Cant.} & \textbf{Costo Est. (COP)} \\
\midrule
\endfirsthead
\multicolumn{4}{c}%
{{\bfseries \tablename\ \thetable{} -- continuación}} \\
\toprule
\textbf{Categoría} & \textbf{Insumo/Material} & \textbf{Cant.} & \textbf{Costo Est. (COP)} \\
\midrule
\endhead
\midrule \multicolumn{4}{r}{{Continúa en la siguiente página}} \\
\endfoot
\bottomrule
\endlastfoot

% Contenido de la tabla
\textbf{Comp. Electrónicos} & Antena parabólica (reacondicionada) & 1 & \$50.000 \\
& LNB (Low-Noise Block) & 1 & \$40.000 \\
& RTL-SDR Dongle & 1 & \$150.000 \\
& Raspberry Pi 4 (o sup.) & 1 & \$250.000 \\
& Tarjeta MicroSD (32 GB) & 1 & \$30.000 \\
& Cables coaxiales y conectores & Varios & \$25.000 \\
& Fuente de alimentación (5V, 3A) & 1 & \$35.000 \\
\addlinespace
\textbf{Mat. Estructurales} & Trípode o base de soporte & 1 & \$70.000 \\
& Caja protectora (intemperie) & 1 & \$20.000 \\
& Tornillería y herrajes & Varios & \$15.000 \\
\addlinespace
\textbf{Mat. Pedagógico} & Impresión de guías y manuales & 1 kit & \$45.000 \\
& Material para talleres & 1 set & \$20.000 \\
\addlinespace
\textbf{Software} & S.O. para Raspberry Pi & 1 & \$0 \\
& Software de radioastronomía & 1 & \$0 \\
\midrule
\textbf{Subtotal Insumos} & & & \textbf{\$750.000} \\
\end{longtable}
\textit{Nota: Los costos son estimaciones y pueden variar. El costo total se
alinea con el precio de venta de \$800.000, dejando un margen para
logística, soporte y gastos administrativos.}

\subsection{Tabla de Recursos Humanos}
El talento humano es el recurso más crítico del proyecto. La siguiente
tabla describe los roles y la dedicación necesaria.

\begin{longtable}{p{3cm} p{4cm} p{2.5cm} p{4cm}}
\caption{Tabla de Recursos Humanos.} \label{tab:rrhh}\\
\toprule
\textbf{Rol} & \textbf{Perfil Requerido} & \textbf{Dedicación} & \textbf{Responsabilidades} \\
\midrule
\endfirsthead
\multicolumn{4}{c}%
{{\bfseries \tablename\ \thetable{} -- continuación}} \\
\toprule
\textbf{Rol} & \textbf{Perfil Requerido} & \textbf{Dedicación} & \textbf{Responsabilidades} \\
\midrule
\endhead
\midrule \multicolumn{4}{r}{{Continúa en la siguiente página}} \\
\endfoot
\bottomrule
\endlastfoot

% Contenido de la tabla
\textbf{Pasante} & Estudiante de últimos semestres de Ing., Física o afines.
& 600h (10 meses) por implementación. & Liderar todo el ciclo: producción,
instalación, capacitación, soporte y documentación. \\
\addlinespace
\textbf{Tutor Académico} & Docente U.D. con experiencia en proyectos STEM.
& 2-4 h/semana. & Supervisar el rigor académico y técnico. Facilitar
recursos. Evaluar informe final. \\
\addlinespace
\textbf{Coordinador} & Profesional con experiencia en gestión de proyectos.
& 4-6 h/semana. & Gestionar relación con colegios, asegurar calidad y
logística. Buscar nuevas alianzas. \\
\addlinespace
\textbf{Docente Contacto} & Docente de ciencias o tecnología del colegio.
& Variable. & Ser el punto de contacto, coordinar estudiantes y actuar como
multiplicador del conocimiento. \\
\end{longtable}

\subsection{Tabla de Maquinaria y Equipo}
La tecnología del proyecto no requiere maquinaria industrial. El equipo
necesario es para ensamblaje y pruebas.

\begin{longtable}{p{3.5cm} p{5cm} p{5cm}}
\caption{Tabla de Maquinaria y Equipo.} \label{tab:maquinaria}\\
\toprule
\textbf{Tipo} & \textbf{Descripción} & \textbf{Uso y Ubicación} \\
\midrule
\endfirsthead
\multicolumn{3}{c}%
{{\bfseries \tablename\ \thetable{} -- continuación}} \\
\toprule
\textbf{Tipo} & \textbf{Descripción} & \textbf{Uso y Ubicación} \\
\midrule
\endhead
\midrule \multicolumn{3}{r}{{Continúa en la siguiente página}} \\
\endfoot
\bottomrule
\endlastfoot

% Contenido de la tabla
\textbf{Herram. Taller} & Cautín, multímetro, alicates, destornilladores.
& Ensamblaje electrónico. Ubic: Laboratorio U.D. / Proyecto. \\
\addlinespace
\textbf{Equipo Cómputo} & Computador portátil.
& Programación, análisis de datos, informes. Ubic: Pasante / Proyecto. \\
\addlinespace
\textbf{Equipo Pruebas} & Osciloscopio, generador de señales (opcional).
& Calibración y verificación del receptor. Ubic: Laboratorio U.D. \\
\addlinespace
\textbf{Transporte} & Automóvil o servicio contratado.
& Traslado de equipo y personal para instalación. Ubic: Pasante / Contratado. \\
\end{longtable}
