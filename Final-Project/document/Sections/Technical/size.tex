\section*{Determinación del Tamaño}

La definición del tamaño del proyecto es un análisis estratégico que balancea la oportunidad de mercado con los recursos disponibles, para asegurar un crecimiento sostenible y minimizar los riesgos. A continuación, se detalla la estrategia de dimensionamiento y se cuantifican las capacidades operativas del proyecto.

\subsection*{Estrategia de Tamaño y Variables Condicionantes}

La estrategia de tamaño seleccionada para este proyecto es la de ``tamaño inferior a la demanda con ampliaciones posteriores''. Esta decisión se fundamenta en un análisis integral de las variables clave que condicionan el proyecto, garantizando una entrada al mercado prudente y una escalabilidad controlada. Las variables analizadas son:

\begin{itemize}
    \item \textbf{Dimensiones del Mercado:} El estudio de mercado ha identificado una demanda potencial robusta, estimada en aproximadamente 270 colegios privados en Bogotá con un perfil alineado a la iniciativa (énfasis en ciencias o astronomía). Esta cifra es sustancialmente mayor que la capacidad operativa inicial, lo que confirma un amplio potencial de crecimiento y justifica una estrategia que no intente satisfacer toda la demanda de inmediato.

    \item \textbf{Capacidad de Financiamiento:} El modelo operativo depende de la venta del producto-servicio, con un costo promedio de \$800.000 COP por unidad, o de la gestión de créditos a corto plazo. Este factor financiero limita la capacidad de producción en la fase inicial, haciendo inviable una operación a gran escala y reforzando la necesidad de un crecimiento gradual y autofinanciado.

    \item \textbf{Tecnología y Disponibilidad de Insumos:} La tecnología del radiotelescopio es de bajo costo y no impone un volumen mínimo de producción para ser viable. Si bien los componentes materiales están disponibles a través de proveedores identificados, el insumo más crítico y limitante es el talento humano. La capacidad real del proyecto está directamente ligada al número de estudiantes pasantes de la Universidad Distrital que puedan ser incorporados cada año.
\end{itemize}

\subsection*{Análisis Cuantitativo de Capacidades}

Para dimensionar correctamente la operación, es necesario desglosar y cuantificar las diferentes capacidades del equipo.

\paragraph{A. Justificación de Carga Horaria y Plan de Trabajo por Pasante.}
Si bien la normativa de la Universidad Distrital establece una dedicación mínima de 384 horas para la pasantía \cite{ud_2022_acuerdo012}, la naturaleza de este proyecto, que funciona como un trabajo de grado integral, demanda una carga horaria superior para garantizar la calidad y el cumplimiento de todos sus objetivos técnicos, pedagógicos y de investigación. Se propone una dedicación de ``600 horas por pasante'' distribuidas a lo largo de dos semestres académicos (aproximadamente 10 meses).

Esta extensión se justifica por la alta complejidad del ``producto-servicio'', que exige no solo la construcción de un dispositivo, sino también la gestión de un proyecto de intervención educativa completo, incluyendo el desarrollo de material didáctico, la capacitación de personal y la evaluación de impacto.

A continuación, se detalla el plan de trabajo y la distribución de las 600 horas a lo largo de los dos semestres:

\vspace{0.5cm} % Espacio vertical

\textbf{SEMESTRE 1: Desarrollo, Producción e Implementación Inicial (Total: 320 horas)}
\begin{itemize}
    \item \textit{Mes 1-2: Planificación y Producción (120 horas).} Foco en el trabajo técnico interno. Incluye la planificación detallada del proyecto, adquisición de componentes, ensamblaje completo del hardware y pruebas iniciales.
    
    \item \textit{Mes 3: Configuración y Preparación Pedagógica (100 horas).} Instalación y configuración de todo el software necesario. Se inicia el desarrollo y la adaptación de las guías didácticas y el material de capacitación para el colegio asignado.
    
    \item \textit{Mes 4-5: Instalación y Capacitación Inicial (100 horas).} Traslado e instalación del equipo en el colegio. Se realiza la calibración final y se ejecutan los primeros talleres de formación intensiva para el núcleo de docentes y el grupo de estudiantes pioneros.
\end{itemize}

\textbf{SEMESTRE 2: Acompañamiento, Soporte y Cierre Académico (Total: 280 horas)}
\begin{itemize}
    \item \textit{Mes 6-7: Acompañamiento de Proyectos (120 horas).} El pasante asesora al grupo de estudiantes en la formulación y ejecución de su primer proyecto de observación real, aplicando lo aprendido en los talleres. Se resuelven dudas y se fomenta la autonomía.
    
    \item \textit{Mes 8: Soporte y Evaluación (80 horas).} Se brinda soporte técnico y pedagógico continuo de manera remota. Se aplican instrumentos para medir el impacto y la apropiación de la tecnología por parte de la comunidad educativa.
    
    \item \textit{Mes 9-10: Cierre y Elaboración de Informe Final (80 horas).} Sistematización de la experiencia, análisis de los datos de impacto y elaboración del documento de trabajo de grado, un requisito académico indispensable que requiere una dedicación significativa.
\end{itemize}

Esta estructura de 600 horas por pasante se considera óptima, ya que asegura que cada fase del proyecto se complete con rigurosidad, permitiendo no solo una implementación técnica exitosa, sino también un acompañamiento pedagógico de alto valor y el cumplimiento de los requisitos académicos del pasante.

\paragraph{B. Capacidad de Formación (Pedagógica).}
La capacidad pedagógica del proyecto está directamente ligada a la implementación en cada institución. Por cada colegio atendido por un pasante, se generará el siguiente impacto formativo directo:
\begin{itemize}
    \item \textbf{Formación Docente:} Capacitación de un núcleo de 2 a 3 docentes por institución, quienes actuarán como multiplicadores del conocimiento.
    
    \item \textbf{Formación Estudiantil:} Capacitación directa de un grupo inicial de 15 a 20 estudiantes (miembros de semilleros o clubes de ciencia), quienes serán los primeros usuarios del equipo.
\end{itemize}

\subsection*{Proyección del Crecimiento Escalonado del Proyecto}
Con base en los cálculos anteriores y la estrategia de crecimiento, se proyecta el tamaño del proyecto para un horizonte de 5 años. A partir del tercer año, se proyecta que la capacidad operativa del proyecto alcance su estado de madurez y se mantenga estática. Esto se debe a que se habrá alcanzado el número máximo de 9 pasantes que el proyecto puede gestionar de manera efectiva en un año educativo, según lo establecido en el estudio de oferta.

\begin{table}[H]
\centering
\caption{Proyección Integrada de Capacidades del Proyecto (Horizonte de 5 Años).}
\label{tab:proyeccion_integrada}
\begin{tabular}{|l|c|c|c|c|c|}
\hline
\textbf{Capacidad} & \textbf{Año 1} & \textbf{Año 2} & \textbf{Año 3} & \textbf{Año 4} & \textbf{Año 5} \\ \hline
\hline
N° de Pasantes & 3 & 6 & 9 & 9 & 9 \\ \hline
Implementación (Colegios) & 3 & 6 & 9 & 9 & 9 \\ \hline
Formación (Docentes) & 6 - 9 & 12 - 18 & 18 - 27 & 18 - 27 & 18 - 27 \\ \hline
Formación (Estudiantes) & 45 - 60 & 90 - 120 & 135 - 180 & 135 - 180 & 135 - 180 \\ \hline
\end{tabular}
\end{table}

De esta manera, el tamaño inicial del proyecto se define como una capacidad operativa para implementar 3 unidades de radiotelescopio en 3 colegios durante el primer año, con la participación de 3 pasantes. Esta escala inicial permitirá validar el modelo, perfeccionar los procesos y construir una base sólida para el crecimiento proyectado en la Tabla \ref{tab:proyeccion_integrada}, asegurando que el proyecto escale de manera sostenible y con un alto estándar de calidad.
