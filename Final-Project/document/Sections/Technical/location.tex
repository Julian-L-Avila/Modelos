\section*{Estudio de Localización Estratégica}

El estudio de localización para este proyecto se enfoca en determinar las áreas geográficas y las instituciones educativas que ofrecen las mayores probabilidades de éxito para la fase piloto. El objetivo es seleccionar un grupo de colegios que no solo tengan una alta afinidad por las ciencias, sino que también estén agrupados geográficamente para optimizar la logística y el soporte.

\subsection*{Fase 1: Macrolocalización por Zonas de Influencia}

Esta fase busca seleccionar la región geográfica más estratégica para enfocar los esfuerzos durante el primer año de implementación. Se consideran no solo las localidades de Bogotá, sino también los municipios aledaños con una alta concentración de colegios de interés.

\paragraph{Alternativas de Macrolocalización.} Se proponen tres zonas de influencia:
\begin{itemize}
    \item \textbf{Alternativa A: Bogotá - Eje Norte (Usaquén y Suba).} Estas localidades concentran un número muy elevado de colegios privados con perfiles bilingües, internacionales y con un demostrado enfoque en programas de ciencia y tecnología.
    \item \textbf{Alternativa B: Municipios Aledaños (Chía y Cota).} Zona de alta densidad de colegios campestres de gran prestigio. Representa un mercado de alto impacto, pero con desafíos logísticos de transporte intermunicipal.
    \item \textbf{Alternativa C: Bogotá - Eje Centro-Occidente (Teusaquillo, Chapinero).} Aunque tiene menos concentración de colegios objetivo, su principal ventaja es la proximidad al centro de operaciones del proyecto (cerca de la Universidad Distrital), lo que facilitaría la gestión diaria.
\end{itemize}

\paragraph{Método de Evaluación.} Se aplica el Método Cualitativo por Puntos, priorizando la afinidad de los colegios con la ciencia como el factor más crítico para asegurar el éxito inicial.

\begin{itemize}
    \item \textbf{Concentración de Colegios Foco-Ciencia (Peso: 40\%):} La calificación se basa en la densidad de colegios privados de alto rendimiento académico y con perfiles bilingües o con énfasis en ciencias. Rankings como los de Sapiens Research (\url{https://www.srg.com.co/}) y directorios de colegios privados muestran una alta concentración de instituciones de este tipo en las localidades de Usaquén y Suba (Eje Norte), y en municipios como Chía. El centro de la ciudad tiene una menor densidad de este perfil específico.
    
    \item \textbf{Potencial de Visibilidad y Prestigio (Peso: 25\%):} Se califica la presencia de colegios ``insignia'' en cada zona, cuyo prestigio podría dar mayor visibilidad al proyecto. Los municipios aledaños (Chía) y el Eje Norte de Bogotá albergan muchos de los colegios con mejor reputación a nivel nacional.
    
    \item \textbf{Facilidad Logística y Tiempo de Viaje (Peso: 20\%):} La calificación se asigna de forma inversa al tiempo de desplazamiento promedio desde el centro de operaciones. Viajes al Eje Norte (Usaquén) pueden tardar entre 30-45 minutos en condiciones normales, mientras que los viajes a municipios como Chía pueden superar la hora, especialmente en picos de tráfico. El Eje Centro tiene la mejor logística por su proximidad.
    
    \item \textbf{Proximidad al Centro de Operaciones (U.D.) (Peso: 15\%):} Este factor es una consecuencia directa del anterior. La zona Centro recibe la máxima calificación por ser la sede del equipo, mientras que los municipios aledaños reciben la más baja por la distancia y tiempos de viaje.
\end{itemize}

\begin{table}[h!]
\centering
\caption{Matriz de Evaluación de Macrolocalización por Zonas de Influencia.}
\label{tab:macro_zonas}
\resizebox{\textwidth}{!}{%
\begin{tabular}{|l|c|>{\centering\arraybackslash}p{2.2cm}|>{\centering\arraybackslash}p{1.2cm}|>{\centering\arraybackslash}p{2.2cm}|>{\centering\arraybackslash}p{1.2cm}|>{\centering\arraybackslash}p{2.2cm}|>{\centering\arraybackslash}p{1.2cm}|}
\hline
\multicolumn{1}{|c|}{\textbf{Factor}} & \textbf{Peso} & \multicolumn{2}{c|}{\textbf{A: Bogotá - Eje Norte}} & \multicolumn{2}{c|}{\textbf{B: Municipios Aledaños}} & \multicolumn{2}{c|}{\textbf{C: Bogotá - Centro}} \\ \hline
 &  & \textbf{Calif. (0-10)} & \textbf{Pond.} & \textbf{Calif. (0-10)} & \textbf{Pond.} & \textbf{Calif. (0-10)} & \textbf{Pond.} \\ \hline
Concentración de Colegios Foco-Ciencia & 40\% & 9 & 3.60 & 8 & 3.20 & 6 & 2.40 \\ \hline
Potencial de Visibilidad y Prestigio & 25\% & 8 & 2.00 & 9 & 2.25 & 7 & 1.75 \\ \hline
Facilidad Logística y Menor Tiempo de Viaje & 20\% & 7 & 1.40 & 5 & 1.00 & 9 & 1.80 \\ \hline
Proximidad al Centro de Operaciones (U.D.) & 15\% & 6 & 0.90 & 4 & 0.60 & 10 & 1.50 \\ \hline
\hline
\textbf{PUNTUACIÓN TOTAL} & \textbf{100\%} & \multicolumn{2}{c|}{\textbf{7.90}} & \multicolumn{2}{c|}{\textbf{7.05}} & \multicolumn{2}{c|}{\textbf{7.45}} \\ \hline
\end{tabular}%
}
\end{table}

El resultado del análisis cuantitativo muestra que la zona de Bogotá - Eje Norte (Alternativa A) obtiene la puntuación más alta (7.90). Aunque el Eje Centro-Occidente ofrece una logística superior por su cercanía al centro de operaciones, la alta concentración de colegios con el perfil exacto que busca el proyecto en el Eje Norte es el factor determinante. Esta zona ofrece el mejor balance entre acceso a un mercado ideal y una logística manejable.

\subsection*{Fase 2: Microlocalización y Selección del Clúster Piloto}

Habiendo seleccionado el Eje Norte de Bogotá como la zona estratégica, la microlocalización consiste en identificar un clúster de 3 colegios geográficamente cercanos y con perfiles de alto potencial para la implementación del primer año. Para ello, se realiza una evaluación cuantitativa de un grupo de colegios candidatos.

Se ha seleccionado un grupo de cinco colegios de alto perfil, todos ubicados en el Eje Norte (Usaquén y Suba). La selección de este grupo se justifica por:
\begin{itemize}
    \item \textbf{Coherencia Geográfica:} Todos se encuentran en la zona de interés, lo que permite optimizar la logística y el soporte.
    \item \textbf{Afinidad Científica:} Muestran evidencia, ya sea por infraestructura, programas existentes (como semilleros) o por su alta reputación académica, de ser un terreno fértil para un proyecto de innovación en ciencias.
    \item \textbf{Alto Potencial de Impacto:} Son instituciones reconocidas, lo que asegura que un caso de éxito en cualquiera de ellas tendrá una gran visibilidad y servirá como un sólido precedente para la expansión del proyecto.
\end{itemize}
Los cinco colegios preseleccionados para la evaluación son: Colegio Calasanz, Colegio Bilingüe Maximino Poitiers, Colegio San Viator \cite{sanviator_web}, Liceo de Cervantes \cite{liceocervantes_web} y Colegio Abraham Lincoln \cite{abrahamlincoln_web}.

\paragraph{Método de Evaluación Cuantitativa.}
Se aplica el Método Cualitativo por Puntos para evaluar y priorizar a los colegios candidatos. Los factores se ponderan según su importancia para asegurar el éxito de la fase piloto del proyecto.

\begin{table}[h!]
\centering
\caption{Matriz de Evaluación de Microlocalización (Selección de Colegios).}
\label{tab:micro_colegios}
\resizebox{\textwidth}{!}{%
\begin{tabular}{|l|c|c|c|c|c|c|c|c|c|c|c|}
\hline
\multicolumn{1}{|c|}{\textbf{Factor}} & \textbf{Peso} & \multicolumn{2}{c|}{\textbf{Calasanz}} & \multicolumn{2}{c|}{\textbf{M. Poitiers}} & \multicolumn{2}{c|}{\textbf{San Viator}} & \multicolumn{2}{c|}{\textbf{Cervantes}} & \multicolumn{2}{c|}{\textbf{A. Lincoln}} \\ \hline
 &  & \textbf{Cal.} & \textbf{Pond.} & \textbf{Cal.} & \textbf{Pond.} & \textbf{Cal.} & \textbf{Pond.} & \textbf{Cal.} & \textbf{Pond.} & \textbf{Cal.} & \textbf{Pond.} \\ \hline
Evidencia de Foco en Ciencias & 40\% & 10 & 4.00 & 7 & 2.80 & 8 & 3.20 & 8 & 3.20 & 8 & 3.20 \\ \hline
Reputación y Visibilidad & 25\% & 8 & 2.00 & 8 & 2.00 & 9 & 2.25 & 10 & 2.50 & 9 & 2.25 \\ \hline
Receptividad Institucional (Est.) & 20\% & 9 & 1.80 & 7 & 1.40 & 8 & 1.60 & 8 & 1.60 & 8 & 1.60 \\ \hline
Ubicación y Facilidad Logística & 15\% & 8 & 1.20 & 9 & 1.35 & 8 & 1.20 & 7 & 1.05 & 7 & 1.05 \\ \hline
\hline
\textbf{PUNTUACIÓN TOTAL} & \textbf{100\%} & \multicolumn{2}{c|}{\textbf{9.00}} & \multicolumn{2}{c|}{\textbf{7.55}} & \multicolumn{2}{c|}{\textbf{8.25}} & \multicolumn{2}{c|}{\textbf{8.35}} & \multicolumn{2}{c|}{\textbf{8.10}} \\ \hline
\end{tabular}%
}
\end{table}

De este modo, la estrategia de localización y selección de aliados queda definida por los resultados del análisis. Se establece como área de enfoque principal el Eje Norte de Bogotá. El ranking cuantitativo permite identificar a los tres colegios con mayor puntuación para conformar el clúster piloto. Los seleccionados son:
\begin{enumerate}
    \item \textbf{Colegio Calasanz (Puntaje: 9.00):} La existencia de un ``Semillero de Astronomía'' activo le otorga la máxima calificación en el factor más importante (Foco en Ciencias), posicionándolo como el candidato ideal.
    \item \textbf{Liceo de Cervantes (Puntaje: 8.35):} Su altísima reputación académica y visibilidad a nivel nacional lo convierten en un socio estratégico de gran impacto.
    \item \textbf{Colegio San Viator (Puntaje: 8.25):} Su prestigio y sólida trayectoria académica en la zona norte garantizan un entorno de excelencia y alta receptividad para el proyecto.
\end{enumerate}
Este enfoque dual permite concentrar los esfuerzos, optimizar la logística y maximizar las probabilidades de éxito al colaborar con las instituciones mejor preparadas y más afines a los objetivos del proyecto.
