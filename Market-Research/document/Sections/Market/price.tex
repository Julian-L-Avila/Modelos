\section{Precio del Producto}%
\label{sec:Precio del Producto}

A continuación se listan los componentes principales del radio telescopio y
un valor estimado de los costos según distribuidor.

Antenas y LNB:
\begin{itemize}
  \item \textbf{Servisistemas:} COP\$ 460.000
  \item \textbf{AZ Colombia Store:} COP\$ 400.000 (precio de cinco antenas)
  \item \textbf{{Mercado Libre Colombia:}} COP \$ 200.000
\end{itemize}

Arduino R3:
\begin{itemize}
  \item \textbf{TD Electrónica:} COP\$ 60.0000
  \item \textbf{Mercado Libre Colombia:} COP\$ 150.000
\end{itemize}

Potenciometro:
\begin{itemize}
  \item \textbf{TD Electrónica:} COP\$ 3.000
  \item \textbf{Mercado Libre Colombia:} COP\$ 4.000
\end{itemize}

Protoboard:
\begin{itemize}
  \item \textbf{TD Electrónica:} COP\$ 20.000
  \item \textbf{Mercado Libre Colombia:} COP\$ 25.000
\end{itemize}

Cables:
\begin{itemize}
  \item \textbf{TD Electrónica:} COP\$ 15.000
  \item \textbf{Vicartechz:} COP\$ 10.000
  \item \textbf{Mercado Libre Colombia:} COP\$ 30.000
\end{itemize}

Se tiene previsto un costo de COP\$ 300.000 para la mano de obra. Haciendo un
promedio de COP\$ 800.000 para la construcción de los radiotelescopios y de
la enseñanza para los estudiantes.
