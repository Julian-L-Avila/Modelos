\section{Financiamiento}%
\label{sec:Financiamiento}

Para la financiar el proyecto, se plantea inicialmente vender los
radiotelescopios a instituciones educativas privadas.
Además del dispositivo, se incluirá la oferta de cursos y capacitaciones
dirigidas a docentes y estudiantes, con el fin de garantizar el uso pedagógico
y científico adecuado del instrumento.
Esta estrategia permitirá consolidar el modelo operativo y asegurar recursos que,
en una fase posterior, posibiliten la implementación del proyecto en
instituciones públicas ubicadas en Bogotá.

A continuación, se detallan las posibles instituciones educativas aliadas clave
para implementar la iniciativa en su fase inicial:

\begin{enumerate}
  \item Colegio Abraham Maslow \cite{maslowweb}:
    Esta institución ha destacado por su participación activa en eventos
    astronómicos.
    Su Club de Astronomía fue galardonado en el concurso internacional
    \#NameExoWorlds, organizado por la Unión Astronómica Internacional.
    El colegio bautizó un sistema planetario de la constelación Antlia con los
    nombres Macondo (estrella) y Melquíades (exoplaneta), en honor a Gabriel
    García Márquez, logrando así un hito en la astronomía y la cultura nacional.

  \item Colegio Empresarial Los Andes \cite{astroandes}:
    La institución cuenta con un planetario y un observatorio llamado
    Astroandes, espacios donde los estudiantes desarrollan actividades de
    observación e investigación astronómica.
    Su enfoque pedagógico está orientado al fomento de la curiosidad científica
    y al aprendizaje activo del universo.

  \item Colegio Padre Manyaneit \cite{manyanetweb}:
    Cuenta con un observatorio astronómico
    denominado ``La estrella de Belén'', desde donde promueven el conocimiento
    del cosmos entre sus estudiantes.
    Se ofrece una actividad extracurricular de astronomía orientada a fomentar
    el amor por el universo y la participación en eventos astronómicos.

  \item Colegio Calasanz Bogotá \cite{calasanzastro}:
    Su Semillero de Astronomía participa activamente en actividades de
    divulgación científica.
    En octubre, recibió la visita del grupo de bioastronomía Shaula en el marco
    de la Semana Mundial del Espacio, donde se realizaron talleres sobre
    satélites artificiales y prototipos creativos, lo cual evidencia su interés
    por fortalecer los espacios de aprendizaje en ciencia.

  \item Gimnasio Bilingüe Campestre Marie Curie \cite{mariecurieweb}:
    Institución educativa privada ubicada en la sabana de Bogotá, que promueve
    el pensamiento crítico, la creatividad y el trabajo colaborativo.
    En su propuesta educativa se destaca el interés por desarrollar habilidades
    del siglo XXI mediante proyectos con impacto en la ciencia, la tecnología y
    el medio ambiente.
\end{enumerate}

Adicionalmente, se identificaron colegios que participaron en el evento AstroFest
\cite{astrofest2023}, organizado por la Universidad Sergio Arboleda,
como posibles aliados interesados en la astronomía: Liceo Campestre Cafam \cite{liceocafamweb},
Colegio Maximino Poitiers \cite{maximinoweb}, Liceo Pedagógico Nuestra Señora
de la Sabiduría \cite{sabiduriaweb} y Colegio Agustiniano Tagaste \cite{agustiniano_tagaste}.

Se realizó una revisión de entidades públicas para evaluar posibles líneas de
financiación y apoyo institucional:

\begin{itemize}
  \item Secretaría de Educación de Bogotá \cite{sedbogota}:
    Aunque actualmente no cuenta con convocatorias abiertas orientadas
    específicamente a proyectos astronómicos, se identificó una experiencia
    previa en 2023 con el programa ``Centro de Interés en Astronomía (CIA)'',
    desarrollado en alianza con el Instituto Distrital de las Artes (Idartes) a
    través del Planetario de Bogotá.
    Esta estrategia STEM benefició a estudiantes de 32 colegios oficiales en 12
    localidades \cite{sed2023}.
    Este antecedente sugiere la posibilidad de recibir apoyo institucional una
    vez el proyecto esté en marcha.

  \item  Atenea (Agencia Distrital para la Educación Superior, la Ciencia y la
    Tecnología): Aunque mantiene convocatorias abiertas, estas están actualmente
    orientadas a temáticas como economía circular, salud y bioeconomía
    \cite{atenea2024}, por lo que no se encontró una línea directa con el
    presente proyecto.

  \item Ministerio de Ciencia, Tecnología e Innovación (Minciencias):
    En su portal se identifica la opción de solicitar donaciones para proyectos
    científicos y de apropiación social del conocimiento; sin embargo, a la fecha
    esta modalidad no se encuentra habilitada y no hay convocatorias
    vigentes alineadas con la iniciativa \cite{minciencias2024}.
\end{itemize}

A pesar de que en el momento actual no se han identificado convocatorias
específicas para proyectos de astronomía escolar, los antecedentes de
colaboración institucional, los programas STEM y la existencia de colegios
con infraestructura y actividades científicas robustas muestran un panorama
favorable para establecer alianzas estratégicas.
La fase piloto en colegios privados permitirá validar la metodología y generar
evidencia del impacto, lo cual facilitaría la posterior búsqueda de
financiamiento y apoyo gubernamental.
