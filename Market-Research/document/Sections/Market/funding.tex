\section{Financiamiento}%
\label{sec:Financiamiento}

Para la financiar el proyecto, se plantea inicialmente vender los
radiotelescopios a instituciones educativas privadas.
Además del dispositivo, se incluirá la oferta de cursos y capacitaciones
dirigidas a docentes y estudiantes, con el fin de garantizar el uso pedagógico
y científico adecuado del instrumento.
Esta estrategia permitirá consolidar el modelo operativo y asegurar recursos que,
en una fase posterior, posibiliten la implementación del proyecto en
instituciones públicas ubicadas en Bogotá.

Para facilitar la adquisición por parte de las instituciones, se proponen dos
modelos de financiación flexibles. El primero consiste en un pago inicial
completo por parte del colegio antes del inicio del proyecto, lo que asegura
los fondos de manera directa. El segundo modelo, pensado para instituciones
que prefieran diferir el desembolso, implica que el proyecto gestione un
crédito a corto plazo con una entidad bancaria, para lo cual se explorarían
líneas de crédito empresarial con bancos de reconocida trayectoria como
\textbf{Bancolombia, Davivienda} o \textbf{Banco de Bogotá}. La institución educativa se
comprometería a realizar el pago al finalizar la implementación, momento en el
cual el proyecto saldaría el préstamo. Esta dualidad busca adaptarse a las
diferentes capacidades y planificaciones presupuestarias de los colegios,
ampliando así el alcance potencial de la iniciativa.

Se realizó una revisión de entidades públicas para evaluar posibles líneas de
financiación y apoyo institucional:

\begin{itemize}
\item Secretaría de Educación de Bogotá \cite{sedbogota}:
Aunque actualmente no cuenta con convocatorias abiertas orientadas
específicamente a proyectos astronómicos, se identificó una experiencia
previa en 2023 con el programa ``Centro de Interés en Astronomía (CIA)'',
desarrollado en alianza con el Instituto Distrital de las Artes (Idartes) a
través del Planetario de Bogotá.
Esta estrategia STEM benefició a estudiantes de 32 colegios oficiales en 12
localidades \cite{sed2023}.
Este antecedente sugiere la posibilidad de recibir apoyo institucional una
vez el proyecto esté en marcha.

\item Atenea (Agencia Distrital para la Educación Superior, la Ciencia y la
Tecnología): Aunque mantiene convocatorias abiertas, estas están actualmente
orientadas a temáticas como economía circular, salud y bioeconomía
\cite{atenea2024}, por lo que no se encontró una línea directa con el
presente proyecto.

\item Ministerio de Ciencia, Tecnología e Innovación (Minciencias):
En su portal se identifica la opción de solicitar donaciones para proyectos
científicos y de apropiación social del conocimiento; sin embargo, a la fecha
esta modalidad no se encuentra habilitada y no hay convocatorias
vigentes alineadas con la iniciativa \cite{minciencias2024}.
\end{itemize}

A pesar de que en el momento actual no se han identificado convocatorias
específicas para proyectos de astronomía escolar, los antecedentes de
colaboración institucional, los programas STEM y la existencia de colegios
con infraestructura y actividades científicas robustas muestran un panorama
favorable para establecer alianzas estratégicas.

La fase piloto en colegios privados permitirá validar la metodología y generar
evidencia del impacto, lo cual facilitaría la posterior búsqueda de
financiamiento y apoyo gubernamental.
