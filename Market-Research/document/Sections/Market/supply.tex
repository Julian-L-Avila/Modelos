\section{Determinación de la Oferta}%
\label{sec:Determinación de la Oferta}

La determinación de la oferta de nuestro proyecto está intrínsecamente
vinculada a las limitaciones en la producción de los radiotelescopios y la
disponibilidad de talento humano.
Específicamente, las principales restricciones son la adquisición y
procesamiento de los materiales de fabricación de cada unidad, así como la
capacidad y dedicación del personal involucrado.
Por ejemplo, la complejidad de ciertos componentes electrónicos y mecánicos,
junto con sus tiempos de entrega, puede influir directamente en el ritmo de
ensamblaje.

En cuanto al talento humano, el proyecto contempla la participación de
estudiantes de los programas de Física o Licenciatura en Física de la
Universidad Distrital Francisco José de Caldas.
Estos estudiantes colaborarán como pasantes, integrando su contribución como
parte de su proceso de trabajo de grado.
Esta modalidad no solo optimiza los recursos, sino que también ofrece una
experiencia formativa valiosa.

Se proyecta contar con la participación de 3 estudiantes durante el primer
año de ejecución del proyecto.
Para el segundo año, se prevé incrementar este número a un total de 6
estudiantes, lo que permitiría una mayor capacidad de producción y soporte.
Es fundamental destacar que todos los estudiantes estarían bajo la supervisión
directa de docentes tutores y adscritos al semillero de Investigación en
Energías Renovables de la Universidad.
Esto asegura un marco de trabajo estructurado y el apoyo académico necesario.
