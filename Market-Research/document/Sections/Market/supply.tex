\section{Determinación de la Oferta}%
\label{sec:Determinación de la Oferta}

La determinación de la oferta de nuestro proyecto está intrínsecamente
vinculada a las limitaciones en la producción de los radiotelescopios y la
disponibilidad de talento humano.
Específicamente, las principales restricciones son la adquisición y
procesamiento de los materiales de fabricación de cada unidad, así como la
capacidad y dedicación del personal involucrado.
Por ejemplo, la complejidad de ciertos componentes electrónicos y mecánicos,
junto con sus tiempos de entrega, puede influir directamente en el ritmo de
ensamblaje.

En cuanto al talento humano, el proyecto contempla la participación de
estudiantes de los programas de Física o Licenciatura en Física de la
Universidad Distrital Francisco José de Caldas.
Estos estudiantes colaborarán como pasantes, integrando su contribución como
parte de su proceso de trabajo de grado.
Esta modalidad no solo optimiza los recursos, sino que también ofrece una
experiencia formativa valiosa.

Se proyecta contar con la participación de 3 estudiantes durante el primer
año de ejecución del proyecto.
Para el segundo año, se prevé incrementar este número a un total de 6
estudiantes, lo que permitiría una mayor capacidad de producción y soporte,
con un ultimo incremento a 9 instituciones, limitando a este valor por ser
la cantidad máxima posible de pasantes que se pueden llegar a tener en un
año educativo.
Es fundamental destacar que todos los estudiantes estarían bajo la supervisión
directa de docentes tutores y adscritos al semillero de Investigación en
Energías Renovables de la Universidad.
Esto asegura un marco de trabajo estructurado y el apoyo académico necesario.

A continuación, se detallan las posibles instituciones educativas aliadas clave
para implementar la iniciativa en su fase inicial:

\begin{enumerate}
  \item Colegio Abraham Maslow \cite{maslowweb}:
    Esta institución ha destacado por su participación activa en eventos
    astronómicos.
    Su Club de Astronomía fue galardonado en el concurso internacional
    \#NameExoWorlds, organizado por la Unión Astronómica Internacional.
    El colegio bautizó un sistema planetario de la constelación Antlia con los
    nombres Macondo (estrella) y Melquíades (exoplaneta), en honor a Gabriel
    García Márquez, logrando así un hito en la astronomía y la cultura nacional.

  \item Colegio Empresarial Los Andes \cite{astroandes}:
    La institución cuenta con un planetario y un observatorio llamado
    Astroandes, espacios donde los estudiantes desarrollan actividades de
    observación e investigación astronómica.
    Su enfoque pedagógico está orientado al fomento de la curiosidad científica
    y al aprendizaje activo del universo.

  \item Colegio Padre Manyaneit \cite{manyanetweb}:
    Cuenta con un observatorio astronómico
    denominado ``La estrella de Belén'', desde donde promueven el conocimiento
    del cosmos entre sus estudiantes.
    Se ofrece una actividad extracurricular de astronomía orientada a fomentar
    el amor por el universo y la participación en eventos astronómicos.

  \item Colegio Calasanz Bogotá \cite{calasanzastro}:
    Su Semillero de Astronomía participa activamente en actividades de
    divulgación científica.
    En octubre, recibió la visita del grupo de bioastronomía Shaula en el marco
    de la Semana Mundial del Espacio, donde se realizaron talleres sobre
    satélites artificiales y prototipos creativos, lo cual evidencia su interés
    por fortalecer los espacios de aprendizaje en ciencia.

  \item Gimnasio Bilingüe Campestre Marie Curie \cite{mariecurieweb}:
    Institución educativa privada ubicada en la sabana de Bogotá, que promueve
    el pensamiento crítico, la creatividad y el trabajo colaborativo.
    En su propuesta educativa se destaca el interés por desarrollar habilidades
    del siglo XXI mediante proyectos con impacto en la ciencia, la tecnología y
    el medio ambiente.
\end{enumerate}

Adicionalmente, se identificaron colegios que participaron en el evento AstroFest
\cite{astrofest2023}, organizado por la Universidad Sergio Arboleda,
como posibles aliados interesados en la astronomía: Liceo Campestre Cafam \cite{liceocafamweb},
Colegio Maximino Poitiers \cite{maximinoweb}, Liceo Pedagógico Nuestra Señora
de la Sabiduría \cite{sabiduriaweb} y Colegio Agustiniano Tagaste \cite{agustiniano_tagaste}.
