La educación en ciencias y tecnología en las instituciones públicas de Bogotá
enfrenta grandes desafíos, entre ellos la falta de infraestructura especializada
que permita a los estudiantes acceder a experiencias prácticas en el campo de la
astronomía. A pesar del creciente interés por la exploración del universo y el
desarrollo de la tecnología aeroespacial, la enseñanza en estos campos sigue
siendo teórica y limitada en recursos que permitan la experimentación directa.

Actualmente, los programas de educación secundaria en ciencias naturales
y física abordan temáticas relacionadas con la astronomía de manera superficial,
sin contar con herramientas que faciliten la observación y el análisis de datos
astronómicos reales. Esto genera una brecha de acceso al conocimiento y una
disminución del interés de los estudiantes por carreras en ciencia, tecnología,
ingeniería y matemáticas (STEM), lo que afecta el desarrollo científico
y tecnológico del país a largo plazo. Además, la astronomía observacional suele
estar restringida a telescopios ópticos, los cuales dependen de condiciones
atmosféricas favorables y presentan limitaciones en la exploración de otros
espectros electromagnéticos. La falta de acceso a tecnologías complementarias,
como los radiotelescopios, impide que los estudiantes puedan explorar nuevas
formas de observación del universo y desarrollar habilidades en el análisis de
datos astronómicos.

Esta carencia de oportunidades para la experimentación científica en la
educación secundaria contribuye a una creciente desconexión de la sociedad
colombiana con la investigación fundamental, generando un desinterés
generalizado en el avance científico. Como consecuencia, se ha reducido el
presupuesto destinado a la investigación básica y, en términos más generales, la
inversión en ciencia y tecnología. Colombia presenta pocas oportunidades para el
desarrollo de la investigación científica, y esta situación se agrava aún más en
disciplinas como la astronomía. La falta de políticas gubernamentales que
prioricen la ciencia y la tecnología ha llevado a que el presupuesto asignado
para estas áreas sea limitado. La astronomía, en particular, ha sido una de las
disciplinas más afectadas por la falta de financiamiento y apoyo institucional.

El impacto de esta problemática se refleja en la escasez de programas educativos
especializados, la carencia de infraestructura adecuada para la investigación
astronómica y la migración de talento científico hacia otros países con mejores
oportunidades. Además, la falta de una cultura científica en la sociedad
colombiana dificulta la generación de vocaciones en disciplinas STEM, lo que
perpetúa el ciclo de desinterés y baja inversión en ciencia.

La ausencia de una estrategia clara para fomentar la investigación científica en
Colombia tiene repercusiones a largo plazo en el desarrollo del país. La ciencia
no solo impulsa la innovación y el crecimiento económico, sino que también
contribuye a la solución de problemas sociales y ambientales. Un país que se
desinteresa por la ciencia es un país que no avanza. La falta de inversión en
investigación y desarrollo limita la capacidad de una nación para competir en el
ámbito global, innovar en sectores estratégicos y mejorar la calidad de vida de
sus ciudadanos. Sin ciencia, no hay progreso tecnológico, no hay desarrollo
sostenible y no hay preparación para enfrentar los desafíos del futuro.

Por ello, es fundamental implementar políticas públicas que promuevan la
investigación en áreas fundamentales como la astronomía y fomentar el interés de
la sociedad en la ciencia mediante estrategias educativas y divulgativas. La
educación científica debe fortalecerse desde la etapa escolar, proporcionando
a los estudiantes herramientas y experiencias que los motiven a explorar el
universo y a desarrollar habilidades analíticas y tecnológicas, claves para el
futuro del país.
