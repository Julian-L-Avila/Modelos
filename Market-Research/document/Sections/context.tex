\begin{itemize}
  \item \textbf{Radioastronomía}: es una rama de la astronomía que estudia los
    objetos celestes en el firmamento, en especial en la Tierra mediante la
    detección, tratamiento y posterior análisis de las ondas de radio.
    A diferencia de la astronomía óptica, la radioastronomía permite observar
    objetos que podrían ser invisibles en el espectro electromagnético visible
    debido a la absorción de la luz por polvo y gas interestelar. Desde el
    descubrimiento de la radioseñal de la Vía Láctea por Karl Jansky en 1931,
    la radioastronomía ha permitido la detección de objetos como pulsares,
    galaxias activas y la radiación de fondo de microondas \cite{Wilson2013}.

  \item \textbf{Radiotelescopio}: es un dispositivo utilizado para el estudio
    de las emisiones de radiación electromagnética de los cuerpos celestes,
    en el rango de las radiofrecuencias \cite{OrdonezToro2018}.

  \item \textbf{Espectro electromagnético}: representa la distribución
    energética del conjunto de las ondas electromagnéticas.
    Se extiende desde radiaciones con longitud de onda pequeña como los rayos
    cósmicos y rayos gamma súper energéticos, hasta ondas con gran longitud
    de onda como las de radio, pasando por la luz ultravioleta, la luz visible
    y los rayos infrarrojos \cite{OrdonezToro2018}.

  \item \textbf{Ondas de radio}: son un tipo de radiación electromagnética,
    que ocupa el rango de las longitudes de onda más grandes del espectro
    electromagnético, comprende desde los 3 kHz de frecuencia, con una longitud
    de onda de 100 km, hasta los 300 GHz de frecuencia, con una longitud de onda
    de 1 mm. Las ondas de radio reciben también el nombre de radiofrecuencia RF,
    se localizan en una parte del espectro electromagnético denominado espectro
    radioeléctrico \cite{Universo2017}.

  \item \textbf{Antena:} Una antena es un dispositivo que convierte señales
    eléctricas en ondas electromagnéticas y viceversa.
    Se utiliza en sistemas de comunicación para emitir o recibir señales de
    radiofrecuencia (RF) \cite{Kraus1950}.

  \item \textbf{Web Scraping:} Proceso proceso automatizado de extraer datos de
    sitios web.
    Implica obtener páginas web, analizar su contenido (HTML, etc.) e identificar
    y seleccionar información específica.
    Los datos extraídos se transforman y almacenan en un formato estructurado
    para su uso posterior \cite{Thapelo2021}.

  \item \textbf{Arduino:} Es una plataforma de hardware libre basada en una placa
    con un microcontrolador y un entorno de desarrollo (IDE) que permite crear
    proyectos interactivos de electrónica y programación \cite{Arduino2025}.

  \item \textbf{Protoboard:} Es una placa de pruebas utilizada para construir
    circuitos electrónicos sin necesidad de soldadura.
    Permite insertar y conectar componentes fácilmente para prototipado rápido
    \cite{Horowitz1990}.

  \item \textbf{LNB:} Es un dispositivo electrónico que se monta en el extremo
    de una antena parabólica.
    Su función principal es captar las señales de microondas de alta frecuencia
    provenientes de un satélite, amplificarlas con el menor ruido posible
    (low noise), y convertirlas a una frecuencia más baja para que puedan ser
    transmitidas a través de un cable coaxial hacia el receptor \cite{ITU2025}.

  \item \textbf{Cable coaxial:} Es un tipo de cable eléctrico diseñado para
    transmitir señales de alta frecuencia con baja pérdida.
    Está compuesto por un conductor central, una capa aislante, un blindaje
    metálico y una cubierta exterior protectora.
    Su diseño concéntrico permite una excelente protección contra interferencias
    electromagnéticas \cite{Horowitz1990}.

\end{itemize}
