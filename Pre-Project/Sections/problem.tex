La educación en ciencias y tecnología en las instituciones públicas de Bogotá
enfrenta grandes desafíos, entre ellos la falta de infraestructura
especializada que permita a los estudiantes acceder a experiencias prácticas en
el campo de la astronomía.
A pesar del creciente interés por la exploración del universo y el desarrollo de
la tecnología aeroespacial, la enseñanza en estos campos sigue siendo teórica y
limitada en recursos que permitan la experimentación directa.

Actualmente, los programas de educación secundaria en ciencias naturales y
física abordan temáticas relacionadas con la astronomía de manera superficial,
sin contar con herramientas que faciliten la observación y el análisis de datos
astronómicos reales.
Esto genera una brecha de acceso al conocimiento y una disminución del interés
de los estudiantes por carreras en ciencia, tecnología, ingeniería y matemáticas,
lo que afecta el desarrollo científico y tecnológico del país a largo plazo.

Además, la astronomía observacional suele estar restringida a telescopios
ópticos, los cuales dependen de condiciones atmosféricas favorables y presentan
limitaciones en la exploración de otros espectros electromagnéticos.
La falta de acceso a tecnologías complementarias, como los radiotelescopios,
impide que los estudiantes puedan explorar nuevas formas de observación del
universo y desarrollar habilidades en el análisis de datos astronómicos.
