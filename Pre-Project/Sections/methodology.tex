Este proyecto se enmarca dentro de una investigación aplicada, la cual busca
generar conocimiento con un propósito práctico y concreto: la implementación de
radiotelescopios en instituciones educativas para mejorar la enseñanza de la
astronomía y fomentar el desarrollo de habilidades científicas en los
estudiantes.

En este caso, la investigación aplicada se centra en evaluar, diseñar
e implementar una infraestructura tecnológica educativa que pueda ser utilizada
directamente en el ámbito académico. Se utilizarán métodos cuantitativos
y cualitativos para analizar la viabilidad del proyecto, su impacto en la
educación y su sostenibilidad a largo plazo.

Para la implementación del proyecto de construcción de radiotelescopios en
instituciones de educación pública en Bogotá, se propone la siguiente
metodología:

\begin{itemize}

\item \textbf{Estudio de mercado:} Los radiotelescopios que se construirán
son de bajo costo, es decir, se busca disminuir al mínimo los costos de la
producción, por esto mismo, si se cuenta con la financiación de diferentes
instituciones de educación privada, se puede ofrecer un radiotelescopio por
institución.

\item \textbf{Estudio técnico:} Para la construcción del radiotelescopio se
necesita una antena parabólica, un arduino R3, un computador con el software
correspondiente para el correcto funcionamiento del arduino, un buscador de
satélites y un circuito amplificado, encargado de aumentar la señal que recibe
la antena para que el arduino pueda leerla.

\item \textbf{Estudio financiero:} Para la antena parabólica, se tiene un precio
promedio de \$500000 COP y un tamaño de 1.2 metros. Para el caso del arduino, su
precio es de \$150000 COP. El computador puede ser proporcionado por la
institución educativa al igual que el software para su funcionamiento, el cual
es de uso libre y sin costo. Para el circuito amplificador se cuenta con un
precio promedio de \$150000 COP. Finalmente, también se debe tener en cuenta el
precio por la capacitación de los docentes para el uso del radiotelescopio y la
mano de obra para construirlo.

\item \textbf{Diagnóstico y selección de instituciones:} Se realizará un estudio
para identificar las instituciones educativas con mayor potencial para albergar
un radiotelescopio, considerando factores como ubicación, infraestructura
disponible y disposición de la comunidad educativa. Esta selección se hará con
base a la base de datos del gobierno:
\url{https://www.datos.gov.co/Educaci-n/LISTADO-COLEGIOS-BOGOTA/qijw-htwa/about_data}

\item \textbf{Diseño y planificación:} Se desarrollará un plan de implementación
detallado con cronogramas, responsables y actividades específicas. Se definirán
las especificaciones de cada radiotelescopio y su integración con el currículo
escolar.
Además, se diseñarán protocolos de operación y seguridad para garantizar su
correcto uso.

\item \textbf{Instalación y puesta en marcha:}  Se realizará la instalación de
los radiotelescopios en los colegios seleccionados, asegurando que su ubicación
minimice interferencias externas y maximice la recepción de señales. Se harán
pruebas de funcionamiento y calibración de los equipos para garantizar su
operatividad óptima.

\item \textbf{Capacitación docente y estudiantil:} Se diseñarán y ejecutarán
programas de formación para docentes y estudiantes, abordando tanto el uso del
radiotelescopio como el análisis e interpretación de datos astronómicos. Se
promoverán proyectos escolares que utilicen el radiotelescopio como herramienta
de aprendizaje.

\item \textbf{Integración curricular:} Se elaborarán guías didácticas
y materiales educativos para facilitar la incorporación del radiotelescopio en
las clases de física, matemáticas y tecnología.
Se desarrollarán metodologías activas que permitan a los estudiantes aplicar
conceptos teóricos en la práctica.

\item \textbf{Monitoreo y evaluación:} Se implementará un sistema de seguimiento
que permita evaluar el impacto del proyecto en la enseñanza de la astronomía
y las ciencias en general.
Se realizarán mediciones periódicas sobre el uso del radiotelescopio, el nivel
de participación estudiantil y la efectividad del aprendizaje.
A partir de estos datos, se harán ajustes para mejorar el proyecto y explorar
su expansión a otras instituciones.

\end{itemize}
