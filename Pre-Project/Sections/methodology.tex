Para la implementación del proyecto de construcción de radiotelescopios en
instituciones de educación pública en Bogotá, se propone la siguiente
metodología:

\begin{itemize}

  \item \textbf{Diagnóstico y selección de instituciones:} Se realizará un
    estudio para identificar las instituciones educativas con mayor potencial
    para albergar un radiotelescopio, considerando factores como ubicación,
    infraestructura disponible y disposición de la comunidad educativa.

  \item \textbf{Diseño y planificación:} Se elaborará un plan detallado que
    incluirá especificaciones técnicas del radiotelescopio, requerimientos de
    espacio y presupuesto. En esta etapa se contará con la asesoría de expertos
    en radioastronomía y educación científica.

  \item \textbf{Adquisición de equipos y materiales:} Se gestionará la compra o
    fabricación de los componentes necesarios para la construcción del
    radiotelescopio, asegurando que sean accesibles y de fácil mantenimiento
    para las instituciones beneficiadas.

  \item \textbf{Instalación y puesta en marcha:} Se llevará a cabo la
    construcción y montaje del radiotelescopio en las instituciones
    seleccionadas, garantizando su correcto funcionamiento y adecuación a las
    necesidades educativas.

  \item \textbf{Capacitación docente y estudiantil:} Se desarrollarán programas
    de formación para docentes y estudiantes en el uso del radiotelescopio,
    análisis de datos astronómicos y aplicación de conocimientos en proyectos
    científicos.

  \item \textbf{Integración curricular:} Se diseñarán estrategias pedagógicas
    para incorporar el uso del radiotelescopio en el currículo escolar,
    fomentando la enseñanza de astronomía, física y matemáticas de manera
    práctica y aplicada.

  \item \textbf{Monitoreo y evaluación:} Se establecerán mecanismos de
    seguimiento y evaluación para medir el impacto del proyecto en la educación
    científica de los estudiantes, identificando oportunidades de mejora y
    expansión a otras instituciones.

\end{itemize}
