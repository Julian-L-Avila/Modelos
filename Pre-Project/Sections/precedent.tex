En los últimos años, diversas iniciativas han buscado integrar la radioastronomía
en entornos educativos, abarcando desde universidades hasta escuelas de nivel
medio y centros de divulgación.
Estos proyectos no solo han facilitado el acceso a estas herramientas de
observación, sino que también han impulsado la adquisición del conocimiento
necesario para el diseño y construcción de radiotelescopios.

\subsection{PARTNeR}

PARTNeR (Proyecto Académico con el Radio Telescopio de NASA en Robledo) es
un programa educativo que permite a estudiantes de secundaria y universidad
operar de manera remota un radiotelescopio de 34 metros de diámetro, ubicado en
el Madrid Deep Space Communications Complex (MDSCC).
Su objetivo es acercar la radioastronomía a las aulas mediante la observación de
sistemas binarios de rayos X, cuásares, la magnetosfera de Júpiter y la cartografía
de radiofuentes en la Vía Láctea.

Desde su inicio en 2003, PARTNeR ha contado con la participación de 85 centros
de educación secundaria, 7 universidades y 6 agrupaciones astronómicas en España.
A lo largo de los años, más de 2,500 estudiantes y 103 profesores han realizado
observaciones, con un total de 105 sesiones científicas.
Además, las actividades presenciales en el Centro de Entrenamiento y Visitantes
han reunido a un promedio de 3,500 estudiantes por curso.

El programa incluye formación docente con cursos a distancia y presenciales,
materiales didácticos y actividades complementarias, como talleres y la revista
científica PARTNeRama.
También ha participado en iniciativas internacionales como Júpiter:
Proyecto 24, una observación continua de 24 horas en colaboración con la NASA y
otros radiotelescopios en EE.UU. y Australia. \parencite{Vaquerizo2010}

\subsection{Del Aula al Universo}

Si bien no se trata de una iniciativa relacionada con radiotelescopios,
\emph{Del Aula al Universo, un telescopio para cada escuela} comparte ideas
similares con el propósito del presente proyecto:
llevar conocimientos en ciencia y tecnología a las instituciones educativas y
fomentar el interés por la investigación científica en los estudiantes.

Esta iniciativa educativa, impulsada por la Benemérita Universidad Autónoma de
Puebla (BUAP), el Instituto Nacional de Astrofísica, Óptica y Electrónica (INAOE)
y Victorinox México, comenzó en 2010 con la construcción y entrega de 100
telescopios newtonianos de 14 cm de diámetro a escuelas de Puebla y Tlaxcala.
Además de dotar a los estudiantes de herramientas para la observación astronómica,
el programa ha impartido capacitación en astronomía observacional y en la
construcción y uso de los telescopios \parencite{AIDCT2011}.

Desde 2011, el programa ha crecido significativamente. Se han construido más de
mil telescopios y han participado más de 5 mil estudiantes y mil profesores de
secundaria y preparatoria en diversas entidades del México, incluyendo Oaxaca,
San Luis Potosí, Veracruz, Morelos, Querétaro, Campeche, Sonora y Quintana Roo.
Para mantener bajos los costos de fabricación, la Facultad de Ciencias Físico
Matemáticas de la BUAP ha desarrollado innovaciones, como el uso de engranes de
lavadoras para construir monturas y permitir la reparación de los telescopios
con piezas accesibles. Además, el programa ha evolucionado hacia un enfoque
educativo más amplio, en el que los docentes aprenden nuevas estrategias para la
enseñanza de la ciencia. \parencite{BoletinesBUAP2021}

Este esfuerzo ha dado lugar a otros proyectos, como la fabricación de microscopios
con materiales reciclados, beneficiando a miles de estudiantes y promoviendo la
exploración científica a nivel escolar.
La iniciativa continúa expandiéndose con el objetivo de acercar la astronomía a
más comunidades y consolidar un impacto duradero en la educación científica del
país.

\subsection{Manuales para la construcción y uso educativo de radiotelescopios}

En la red se encuentra una gran cantidad de documentación relacionada con la
construcción de radiotelescopios y actividades afines para su uso docente.
Estos materiales permiten acercar el conocimiento en ciencia y tecnología a las
instituciones educativas, facilitando la enseñanza de la radioastronomía y la
astrofísica.
A continuación, se presentan algunos manuales relevantes en este campo.

El \emph{Instituto Nacional de Astrofísica, Óptica y Electrónica} (INAOE) ha
desarrollado un manual para la construcción de radiotelescopios caseros de bajo
costo en la banda de 12 GHz con fines educativos. \parencite{AbrahamLuna2021}
Este manual detalla los componentes necesarios para la construcción del
radiotelescopio y propone diversas prácticas para su uso en la divulgación
científica y la enseñanza.

El observatorio ALMA (Atacama Large Millimeter/submillimeter Array) es un
conjunto de radiotelescopios ubicado en el desierto de Atacama, Chile, operado
en conjunto por el \emph{European Southern Observatory} (ESO),
el \emph{National Astronomical Observatory of Japan} (NAOJ) y
el \emph{National Radio Astronomy Observatory} (NRAO).

En colaboración con estas instituciones, se creó el manual
\emph{Radioastronomía ALMA en la Escuela}, dirigido a docentes interesados en
ampliar sus conocimientos sobre radioastronomía y sobre el funcionamiento del
observatorio ALMA. \parencite{Gallardo2021}

Este documento está estructurado en cuatro capítulos:
\begin{enumerate}
  \item Historia y principios generales de la radioastronomía.
  \item Fundamentos físicos de la radioastronomía, incluyendo fenómenos como
    refracción, reflexión y poder de resolución.
  \item Líneas de investigación en radioastronomía que pueden abordarse con ALMA.
  \item Actividades didácticas organizadas por nivel de dificultad para su
    integración en clases de ciencias naturales.
\end{enumerate}

El manual está diseñado para docentes con conocimientos básicos de física,
química y álgebra, e incluye un glosario de términos científicos para facilitar
su comprensión.

En un contexto más cercano, la \emph{Universidad Pedagógica Nacional de Colombia}
ha desarrollado un manual detallado sobre el diseño y construcción de un
radiotelescopio de bajo costo. \parencite{Penaloza2023}

Este documento describe:

\begin{itemize}
  \item Diseño del radiotelescopio y sus módulos constituyentes.
  \item Componentes y costos de los materiales utilizados.
  \item Procedimientos de conexión, uso y calibración del radiotelescopio.
  \item Actividades experimentales basadas en un enfoque constructivista y
    enseñanza por investigación guiada.
\end{itemize}

Se concluye que es posible construir un radiotelescopio pequeño y preciso con
materiales accesibles, brindando una oportunidad para que estudiantes de nivel
secundario participen en observaciones científicas y desarrollen conocimientos
en radioastronomía y física.

\subsection{ESAOBELA: Formación en Astronomía Observacional para Estudiantes Latinoamericanos}

Como parte de los esfuerzos por fortalecer la educación en astronomía en América
Latina, se han desarrollado diversas iniciativas para acercar a los estudiantes
a la observación y estudio del universo.
Entre ellas, la Escuela de Astronomía Observacional para Estudiantes
Latinoamericanos (ESAOBELA) ofrece un curso integral de astronomía, incluyendo
la radioastronomía, con la participación de investigadores de diversas
instituciones mexicanas. \parencite{GobiernoMexico2024}

Organizado por el Instituto Nacional de Astrofísica, Óptica y Electrónica (INAOE)
y el Instituto de Astronomía de la Universidad Nacional Autónoma de México (IA-UNAM),
ESAOBELA reúne a estudiantes de distintos países latinoamericanos, brindándoles
formación teórica y práctica en astronomía.
El programa combina clases magistrales con sesiones observacionales en el
telescopio del Observatorio Astronómico Nacional (OAN), abordando temas como
astronomía de posición, evolución estelar, instrumentación astronómica,
radioastronomía y astronomía en distintas longitudes de onda.

La inclusión de la radioastronomía en el currículo de ESAOBELA resalta la
importancia de este campo en la educación astronómica de la región, abriendo la
posibilidad de introducirlo a estudiantes más jóvenes mediante materiales y
proyectos educativos adaptados.

\subsection{ El Congreso Colombiano de Astronom\'ia y Astrof\'isica}

En eventos como el Congreso Colombiano de Astronom\'ia y Astrof\'isica, se
fomenta el intercambio de conocimientos entre estudiantes, docentes e
investigadores, brindando un espacio para la presentaci\'on de proyectos
innovadores en el campo de la astronom\'ia.

Durante la edici\'on del a\~no pasado, estudiantes de la Universidad Distrital
presentaron un radiotelescopio desarrollado a lo largo de un semestre.
A partir de esta experiencia, surgieron propuestas para llevar esta tecnolog\'ia
a diferentes sectores educativos, con el objetivo de divulgar la astronom\'ia y
promover el conocimiento sobre esta disciplina. \parencite{Anzola2024}

Esta iniciativa busca facilitar el acceso a la radioastronom\'ia mediante el
dise\~no de radiotelescopios caseros, fomentando el aprendizaje pr\'actico y la
experimentaci\'on en las aulas.
Adem\'as, permite integrar la astronom\'ia en el curr\'iculo escolar y motivar a
los estudiantes a explorar el universo desde una perspectiva cient\'ifica.
