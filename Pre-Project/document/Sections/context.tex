
\begin{itemize}
    \item \textbf{Radioastronomía}: es una rama de la astronomía que estudia los objetos celestes
    en el firmamento, en especial en la Tierra mediante la detección, tratamiento
    y posterior análisis de las ondas de radio.
    A diferencia de la astronomía óptica, la radioastronomía permite observar
    objetos que podrían ser invisibles en el espectro electromagnético visible
    debido a la absorción de la luz por polvo y gas interestelar. Desde el
    descubrimiento de la radioseñal de la Vía Láctea por Karl Jansky en 1931, la
    radioastronomía ha permitido la detección de objetos como pulsares, galaxias
    activas y la radiación de fondo de microondas \cite{wilson2013radioastronomy}.

    \item \textbf{Radiotelescopio}: es un dispositivo utilizado para el estudio de las emisiones de
    radiación electromagnética de los cuerpos celestes, en el rango de las radiofrecuencias 
    \cite{ordonez2018radiotelescopio}.
\end{itemize}



