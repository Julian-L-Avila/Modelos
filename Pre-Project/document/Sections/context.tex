La radioastronomía es una rama de la astronomía que estudia los objetos celestes
en el firmamento, en especial en la tierra mediante la detección, tratamiento
y posterior análisis de las ondas de radio.
A diferencia de la astronomía óptica, la radioastronomía permite observar
objetos que podrían ser in- visibles en el espectro electromagnético visible
debido a la absorción de la luz por polvo y gas interestelar. Desde el
descubrimiento de la radioseñal de la Vía Láctea por Karl Jansky en 1931, la
radioastronomía ha permitido la detección de objetos como pulsares, galaxias
activas y la radiación de fondo de microondas.
