
\begin{itemize}
    \item \textbf{Radioastronomía}: es una rama de la astronomía que estudia los objetos celestes
    en el firmamento, en especial en la Tierra mediante la detección, tratamiento
    y posterior análisis de las ondas de radio.
    A diferencia de la astronomía óptica, la radioastronomía permite observar
    objetos que podrían ser invisibles en el espectro electromagnético visible
    debido a la absorción de la luz por polvo y gas interestelar. Desde el
    descubrimiento de la radioseñal de la Vía Láctea por Karl Jansky en 1931, la
    radioastronomía ha permitido la detección de objetos como pulsares, galaxias
    activas y la radiación de fondo de microondas \cite{wilson2013radioastronomy}.

    \item \textbf{Radiotelescopio}: es un dispositivo utilizado para el estudio de las emisiones de
    radiación electromagnética de los cuerpos celestes, en el rango de las radiofrecuencias 
    \cite{ordonez2018radiotelescopio}.

    \item \textbf{Espectro electromagnético}: representa la distribución energética del conjunto de las
    ondas electromagnéticas. Se extiende desde radiaciones con longitud de onda pequeña
    como los rayos cósmicos y rayos gamma súper energéticos, hasta ondas con gran longitud
    de onda como las de radio, pasando por la luz ultravioleta, la luz visible y los rayos
    infrarrojos \cite{ordonez2018radiotelescopio}.

    \item \textbf{Ondas de radio}: son un tipo de radiación electromagnética, que ocupa el rango de
    las longitudes de onda más grandes del espectro electromagnético, comprende desde
    los 3 kHz de frecuencia, con una longitud de onda de 100 km, hasta los 300 GHz de
    frecuencia, con una longitud de onda de 1 mm. Las ondas de radio reciben también el
    nombre de radiofrecuencia RF, se localizan en una parte del espectro electromagnético
    denominado espectro radioeléctrico \cite{ventanas2017ondas}.

    
\end{itemize}



