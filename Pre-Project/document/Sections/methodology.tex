Este proyecto se enmarca dentro de una investigación aplicada, la cual busca    
generar conocimiento con un propósito práctico y concreto: la implementación de 
radiotelescopios en instituciones educativas para mejorar la enseñanza de la    
astronomía y fomentar el desarrollo de habilidades científicas en los           
estudiantes.                                                                    

En este caso, la investigación aplicada se centra en evaluar, diseñar           
e implementar una infraestructura tecnológica educativa que pueda ser utilizada 
directamente en el ámbito académico. Se utilizarán métodos cuantitativos        
y cualitativos para analizar la viabilidad del proyecto, su impacto en la       
educación y su sostenibilidad a largo plazo.                                    

Inicialmente, los radiotelescopios serán instalados en colegios privados, ya    
que estas instituciones pueden colaborar con la financiación del proyecto.      
Además, suelen contar con  mayor flexibilidad administrativa, lo cual facilita  
la implementación inicial y la evaluación de los resultados. Esta estrategia    
permitirá consolidar la infraestructura y los procesos educativos necesarios,   
con miras a expandir posteriormente la iniciativa hacia los colegios públicos   
de Bogotá, garantizando así un mayor alcance e inclusión educativa.
