\documentclass[11pt,letterpaper]{article}
\usepackage[utf8]{inputenc}
\usepackage[spanish]{babel}
\usepackage{geometry}
\usepackage{fancyhdr}
\usepackage{booktabs}
\usepackage{longtable}
\usepackage{array}
\usepackage{enumitem}
\usepackage{tikz}
\usepackage{hyperref}
\usepackage{ragged2e}
\usepackage{amsmath}
\usepackage{pdflscape}
\usepackage{tabularx}
\usepackage{caption}
\usepackage{siunitx}
\usepackage[spanish]{cleveref}

\geometry{
  letterpaper,
  left=1in,
  right=1in,
  top=1in,
  bottom=1in
}

\pagestyle{fancy}
\fancyhf{}
\fancyhead[L]{\textit{Gestión de Costos}}
\fancyfoot[C]{\thepage}
\renewcommand{\headrulewidth}{0.4pt}
\renewcommand{\footrulewidth}{0.4pt}

\usepackage{pgfplots}
\usepackage{pgfplotstable}
\pgfplotsset{compat=1.18}
\usetikzlibrary{backgrounds, trees, shadows, arrows.meta}

\title{\textbf{Propuesta de diseño de una aplicación móvil para gestionar la información de consumo de energía en los hogares de Bogotá} \\ [1em] \large Gestión de Costos}
\author{
  Laura Herrera \\
  Julian Avila \\
  Bryan Martinez
}
\date{\today}

\begin{document}

\maketitle
\thispagestyle{empty}

\newpage
\tableofcontents
\newpage

\section*{CONTROL DE VERSIONES}

\begin{table}[h!]
  \centering
  \begin{tabularx}{\textwidth}{@{}lllllX@{}}
    \toprule
    \textbf{Versión} & \textbf{Hecha por} & \textbf{Revisada por} & \textbf{Aprobada por} & \textbf{Fecha} & \textbf{Motivo} \\ \midrule
    1.0 & Grupo 2 & Grupo 2 & Grupo 2 & 16/11/2025 & Versión Inicial \\ \bottomrule
  \end{tabularx}
\end{table}

\begin{table}[h!]
  \centering
  \begin{tabularx}{\textwidth}{@{}Xl@{}}
    \toprule
    \textbf{NOMBRE DEL PROYECTO} & \textbf{SIGLAS DEL PROYECTO} \\ \midrule
    Diseño de una Aplicación Móvil para la Gestión de Consumo de Energía en Hogares de Bogotá. & (No posee.) \\ \bottomrule
  \end{tabularx}
\end{table}

\section{PLANIFICACIÓN DE LA GESTIÓN DE COSTOS DEL PROYECTO}

La gestión de costos del proyecto se articula mediante un plan que
especifica el método y los criterios para planificar, estructurar y
controlar los costos. Este documento constituye dicho plan.

\subsection{Costo del Ciclo de Vida del Proyecto}

El análisis del costo del ciclo de vida considera el impacto de las
decisiones actuales del proyecto sobre los costos operativos futuros
(utilización, mantenimiento y soporte del producto).

Para este proyecto, el costo del ciclo de vida abarca:

\begin{itemize}
  \item Costos del Proyecto (Inversión): Recursos necesarios para el
    diseño, desarrollo y entrega del prototipo funcional. La Línea Base
    de Costos para el primer año se establece en
    \textbf{\$309,500,000 COP}.
  \item Costos Operacionales Recurrentes: Costos asociados con el uso
    y mantenimiento futuro de la aplicación y la infraestructura una
    vez que el proyecto concluya.
  \item Beneficios del Ciclo de Vida: Ahorros o retornos previstos
    que justifican la inversión, como la minimización de la
    incertidumbre sobre el consumo y el fomento del pago oportuno.
\end{itemize}

\subsection{Tipos de Costos}

Los costos del proyecto se clasifican para su análisis y control
conforme a las categorías del \cref{tab:classification-costs}.

\begin{table}[htpb!]
  \centering
  \begin{tabularx}{\textwidth}{@{}lX@{}}
    \toprule
    \textbf{Clasificación} & \textbf{Descripción para el Proyecto} \\
    \midrule
    Costos Directos &
    Costos atribuibles directamente a una actividad específica del
    proyecto (ej. mano de obra del desarrollador, adquisición de
    licencias de software y servidores). \\
    \addlinespace
    Costos Indirectos &
    Costos no asignables a un paquete de trabajo único, sino
    distribuidos (ej. gastos administrativos, alquiler de oficinas,
    servicios, sueldo del Gerente del Proyecto). \\
    \addlinespace
    Costos Fijos &
    Costos que no varían con el volumen de producción del proyecto
    (ej. salario del equipo de gestión, alquiler de equipos). El costo
    de RRHH (\$12,600,000 COP mensuales) se considera fijo. \\
    \addlinespace
    Costos Variables &
    Costos que fluctúan en proporción a la cantidad de trabajo
    realizado (ej. servicios de consultoría por demanda, costos de
    procesamiento de datos por volumen). \\
    \bottomrule
  \end{tabularx}
  \captionof{table}{Clasificación de Costos del Proyecto}
  \label{tab:classification-costs}
\end{table}

Las categorías principales de gasto identificadas son: Personal, Equipos
(hardware/software), Servicios, Documentación y Administración.

\subsection{Tipos de Estimaciones}

La estimación de costos aproxima los recursos monetarios requeridos
para completar las actividades del proyecto. Se emplean los métodos del
\cref{tab:methods}.

\begin{table}[htpb!]
  \centering
  \begin{tabularx}{\textwidth}{@{}lX@{}}
    \toprule
    \textbf{Tipo de Estimación} & \textbf{Aplicación al Proyecto} \\
    \midrule
    Estimación Análoga (Descendente) &
    Utilizada en fases iniciales, basada en datos históricos de
    proyectos similares (ej. desarrollo de apps). \\
    \addlinespace
    Estimación Paramétrica &
    Utiliza modelos matemáticos y datos históricos (ej. costo por
    punto de función o por línea de código). \\
    \addlinespace
    Estimación Ascendente &
    Requiere la descomposición de paquetes de trabajo (EDT Nivel 4)
    en actividades, sumando sus costos para obtener el total. \\
    \addlinespace
    Estimación por Tres Valores (PERT) &
    Calcula un costo esperado ($cE$) basado en el costo Optimista
    ($cO$), Más Probable ($cM$) y Pesimista ($cP$).
    Este enfoque se alinea con el usado para la estimación de
    duraciones en el cronograma. \\
    \bottomrule
  \end{tabularx}
  \captionof{table}{Tipos de Estimaciones de Costos}
  \label{tab:methods}
\end{table}

\subsection{Precisión de las Estimaciones}

La precisión de las estimaciones aumenta conforme avanza el proyecto y se
dispone de más información:
\begin{itemize}
  \item Orden de Magnitud (ROM): En fases iniciales, con un rango de
    $-25\%$ a $+75\%$.
  \item Presupuesto Definitivo: En fases posteriores, ajustado a un
    rango de $-5\%$ a $+10\%$.
\end{itemize}

\section{DETERMINACIÓN DEL PRESUPUESTO}

La determinación del presupuesto agrega las estimaciones de costos de
las actividades y paquetes de trabajo para establecer la Línea Base de
Costos autorizada, que servirá como referencia para el control del
proyecto.

\subsection{Estructura del Presupuesto y Línea Base de Costos}

El presupuesto total del proyecto se compone de:
\begin{enumerate}
  \item Estimación de Costos de las Actividades: Suma de los costos
    asociados a los paquetes de trabajo de la EDT.
  \item Reserva para Contingencias: Presupuesto asignado para gestionar
    los riesgos identificados (conocidos-desconocidos).
\end{enumerate}

La suma de (1) y (2) conforma la \textbf{Línea Base de Costos}.
Esta es la referencia aprobada contra la cual se mide el desempeño.

\begin{enumerate}
  \setcounter{enumi}{2} % Continuar la numeración
  \item Reserva de Gestión: Presupuesto para trabajo no previsto
    (desconocidos-desconocidos) que está dentro del alcance.
\end{enumerate}

La suma de la Línea Base de Costos (1+2) más la Reserva de Gestión (3)
conforma el \textbf{Presupuesto Total del Proyecto}.

\subsection{Línea Base de Costos del Proyecto}

Para el Proyecto de la App, la Línea Base de Costos aprobada para el
primer año es:

\begin{center}
  \$309,500,000 COP
\end{center}

Este valor se utiliza como referencia para monitorear y controlar el
desempeño del proyecto.

\section{FLUJO DE CAJA DEL PROYECTO}

Los Requisitos de Financiamiento del Proyecto (Flujo de Caja) se derivan
de la Línea Base de Costos y representan los desembolsos proyectados
a lo largo del tiempo.

La actividad de Proyección de Ingresos y Flujo de Caja es un entregable
clave para el análisis financiero y la determinación de la viabilidad
económica (TIR y VPN). El resumen del presupuesto que compone la línea
base se presenta en el \cref{tab:sumary}.

\begin{table}[h!]
  \centering
  \begin{tabular}{l S[table-format=9.0]}
    \toprule
    \textbf{Categoría de Costo (Año 1)} & {\textbf{Presupuesto (COP)}} \\
    \midrule
    Inversión Inicial (Aporte Socios) & 80500000 \\
    Costos Operativos (Nómina, Arriendo, Servicios) & 229000000 \\
    Costos de Inversión (Equipos, Licencias) & 75500000 \\
    \midrule
    \textbf{PRESUPUESTO TOTAL ESTIMADO (AÑO 1)} & \textbf{309500000} \\
    \bottomrule
  \end{tabular}
  \caption{Resumen del Presupuesto (Año 1)}
  \label{tab:sumary}
\end{table}

\section{PLANIFICACIÓN DEL CONTROL Y SEGUIMIENTO}

El control de costos monitorea el estado del proyecto para mantener los
gastos dentro de la línea base aprobada (\$309.5M) y gestionar los
cambios.

La herramienta principal para el control es la Gestión del Valor Ganado
(EVM), que integra el alcance, el cronograma y los costos. El EVM
proporciona métricas de desempeño mediante la comparación de las
siguientes tres variables:

\begin{itemize}
  \item Valor Planificado (PV): El presupuesto autorizado para el
    trabajo programado.
  \item Costo Real (AC): El costo real incurrido por el trabajo
    realizado.
  \item Valor Ganado (EV): El valor del trabajo completado (medido en
    términos del presupuesto).
\end{itemize}

El desempeño de costos se mide con el Índice de Desempeño del Costo
(CPI = EV / AC). Un CPI inferior a 1.0 indica una desviación negativa:
los gastos son mayores a los planificados para el trabajo realizado.
Las desviaciones significativas requerirán un análisis de reservas y
podrían generar Solicitudes de Cambio formales para ajustar la Línea
Base de Costos.

\end{document}
