\documentclass[11pt,letterpaper]{article}
\usepackage[utf8]{inputenc}
\usepackage[spanish]{babel}
\usepackage{geometry}
\usepackage{fancyhdr}
\usepackage{booktabs}
\usepackage{longtable}
\usepackage{array}
\usepackage{enumitem}
\usepackage{tikz}
\usepackage{hyperref}
\usepackage{ragged2e}
\usepackage{amsmath}
\usepackage{pdflscape}
\usepackage{tabularx}
\usepackage{caption}

\geometry{
  letterpaper,
  left=1in,
  right=1in,
  top=1in,
  bottom=1in
}

\pagestyle{fancy}
\fancyhf{}
\fancyhead[L]{\textit{Gestión del Alcance y Cronograma del Proyecto}}
\fancyfoot[C]{\thepage}
\renewcommand{\headrulewidth}{0.4pt}
\renewcommand{\footrulewidth}{0.4pt}

\usepackage{pgfplots}
\usepackage{pgfplotstable}
\pgfplotsset{compat=1.18}
\usetikzlibrary{backgrounds, trees, shadows, arrows.meta}

\title{\textbf{Propuesta de diseño de una aplicación móvil para gestionar la información de consumo de energía en los hogares de Bogotá} \\ [1em] \large Gestión del Alcance, EDT, PERT y CPM}
\author{
  Laura Herrera \\
  Julian Avila \\
  Bryan Martinez
}
\date{\today}

\begin{document}

\maketitle
\thispagestyle{empty}

\newpage
\tableofcontents
\newpage

\section*{CONTROL DE VERSIONES}

\begin{table}[h!]
  \centering
  \begin{tabularx}{\textwidth}{@{}lllllX@{}}
    \toprule
    \textbf{Versión} & \textbf{Hecha por} & \textbf{Revisada por} & \textbf{Aprobada por} & \textbf{Fecha} & \textbf{Motivo} \\ \midrule
    1.0 & Grupo 2 & Grupo 2 & Grupo 2 & 16/11/2025 & Versión Inicial \\ \bottomrule
  \end{tabularx}
\end{table}

\begin{table}[h!]
  \centering
  \begin{tabularx}{\textwidth}{@{}Xl@{}}
    \toprule
    \textbf{NOMBRE DEL PROYECTO} & \textbf{SIGLAS DEL PROYECTO} \\ \midrule
    Diseño de una Aplicación Móvil para la Gestión de Consumo de Energía en Hogares de Bogotá. & (No posee.) \\ \bottomrule
  \end{tabularx}
\end{table}

\section{PLANIFICACIÓN DE LA GESTIÓN DE COSTOS DEL PROYECTO}

El proceso Planificar la Gestión de los Costos define cómo se planificarán, estructurarán y controlarán los costos a lo largo del proyecto. El resultado de este proceso es este Plan de Gestión de los Costos, el cual es un componente del Plan para la Dirección del Proyecto.

\subsection{Costo del Ciclo de Vida del Proyecto}

El Plan de Gestión de Costos debe considerar el efecto de las decisiones del proyecto sobre los costos recurrentes posteriores de utilizar, mantener y dar soporte al producto.

Para el proyecto de la App Móvil, el costo del ciclo de vida abarca:

\begin{itemize}
  \item \textbf{Costos del Proyecto (Inversión):} Los costos necesarios para el diseño, desarrollo y entrega del prototipo funcional y la documentación. La Línea Base de Costos para el primer año se ha establecido en \textbf{\$309,500,000 COP}.
  \item \textbf{Costos Operacionales Recurrentes:} Los costos asociados con el uso y mantenimiento futuro de la aplicación y la infraestructura una vez que el proyecto de diseño y desarrollo concluya. El entregable "Mantenimiento del Aplicativo (Plan)" (1.2.4) define estos costos posteriores.
  \item \textbf{Beneficios del Ciclo de Vida (Ahorro Previsto):} Aunque la gestión de costos se centra en los recursos para completar el proyecto, el contexto implica beneficios a largo plazo, como la minimización de la incertidumbre sobre el consumo y el aumento del pago oportuno de las facturas en CODENSA, justificando la inversión.
\end{itemize}

\subsection{Tipos de Costos}

De acuerdo con el modelo de referencia, los costos se clasifican de la siguiente manera para el proyecto de la App:

\begin{table}[h!]
  \centering
  \begin{tabularx}{\textwidth}{@{}lX@{}}
    \toprule
    \textbf{Clasificación de Costos} & \textbf{Descripción para el Proyecto de la App} \\ \midrule
    \textbf{Costos Directos} & Costos que pueden ser directamente atribuidos a una actividad, como la mano de obra del equipo (Ingeniero desarrollador) o la adquisición de licencias de software y servidores. \\ \addlinespace
    \textbf{Costos Indirectos} & Costos no asignables directamente a un paquete de trabajo, sino distribuidos entre varios (Gastos administrativos, alquiler de oficinas, servicios generales, sueldo del Gerente del Proyecto). \\ \addlinespace
    \textbf{Costos Fijos} & Costos que no varían con el volumen de producción del proyecto (Ej., el salario del equipo de gestión, o el alquiler de equipos). El costo de RRHH (\$12,600,000 COP mensuales) podría considerarse fijo. \\ \addlinespace
    \textbf{Costos Variables} & Costos que varían con la cantidad de trabajo realizado (Ej., servicios de consultoría externos por demanda o costos de procesamiento de datos por volumen). \\ \bottomrule
  \end{tabularx}
  \captionof{table}{Clasificación de Costos del Proyecto}
\end{table}

Las categorías principales de gasto identificadas son: Personal (recurso humano), Equipos (hardware/software), Servicios (internet, energía), Documentación y Administración.

\subsection{Tipos de Estimaciones}

El proceso Estimar los Costos (7.2) desarrolla una aproximación de los recursos monetarios necesarios. Se utiliza una combinación de métodos:

\begin{table}[h!]
  \centering
  \begin{tabularx}{\textwidth}{@{}lX@{}}
    \toprule
    \textbf{Tipo de Estimación (PMBOK / Modelo Base)} & \textbf{Aplicación al Proyecto de la App} \\ \midrule
    \textbf{Estimación Análoga (Descendente)} & Se utiliza en las fases iniciales cuando hay poca información. Se basa en información histórica de proyectos similares de desarrollo de apps. \\ \addlinespace
    \textbf{Estimación Paramétrica} & Utiliza modelos matemáticos y datos históricos (por ejemplo, costo por línea de código, o costo por punto de función). \\ \addlinespace
    \textbf{Estimación Ascendente} & Requiere descomponer los paquetes de trabajo (EDT Nivel 4) en actividades, sumando estos costos para obtener el total del proyecto. \\ \addlinespace
    \textbf{Estimación por Tres Valores (PERT)} & Utiliza el cálculo de un costo esperado ($cE$) basado en el costo Optimista ($cO$), Más Probable ($cM$), y Pesimista ($cP$). El proyecto ya aplica este enfoque para estimar la \textit{duración} (ver documento de Alcance). \\ \bottomrule
  \end{tabularx}
  \captionof{table}{Tipos de Estimaciones de Costos}
\end{table}

\subsection{Precisión de las Estimaciones}

Las estimaciones de costos se revisarán a lo largo del proyecto, aumentando su exactitud:
\begin{itemize}
  \item \textbf{Orden de Magnitud (ROM):} En las fases iniciales, con un rango de $-25\%$ a $+75\%$.
  \item \textbf{Presupuesto Definitivo:} A medida que se obtiene más información, se ajusta a un rango más estrecho de $-5\%$ a $+10\%$.
\end{itemize}

\section{DETERMINACIÓN DEL PRESUPUESTO}

El proceso Determinar el Presupuesto (7.3) suma las estimaciones de costos de las actividades/paquetes de trabajo para establecer la Línea Base de Costos autorizada.

\subsection{Estructura del Presupuesto y Línea Base de Costos}

El presupuesto del proyecto de la aplicación se construye sumando:
\begin{enumerate}
  \item \textbf{Estimación de Costos de las Actividades:} Costos asociados a los paquetes de trabajo de la EDT (Nivel 4), como el Cálculo de Costos (1.3.1.1) y la Definición de Requisitos (1.1.2.1).
  \item \textbf{Reserva para Contingencias:} Presupuesto asignado a los riesgos identificados (conocidos-desconocidos) que son aceptados.
\end{enumerate}

La suma de (1) y (2) conforma la \textbf{Línea Base de Costos}.

\begin{enumerate}
  \setcounter{enumi}{2} % Continuar la numeración
  \item \textbf{Reserva de Gestión:} Presupuesto para trabajo no previsto (desconocidos-desconocidos) que está dentro del alcance.
\end{enumerate}

La suma de la Línea Base de Costos más (3) conforma el \textbf{Presupuesto Total del Proyecto}.

\subsection{Línea Base de Costos del Proyecto (HeredíaRamírez)}

Para el Proyecto de la App (HeredíaRamírez), la \textbf{Línea Base de Costos} aprobada para el primer año se ha determinado en:

\begin{center}
  \textbf{\$309,500,000 COP}
\end{center}

Este valor se utiliza como referencia para monitorear y controlar el desempeño del proyecto.

\section{FLUJO DE CAJA DEL PROYECTO}

El Flujo de Caja (o Requisitos de Financiamiento del Proyecto) se deriva de la Línea Base de Costos y representa los costos proyectados que se pagarán a lo largo del tiempo.

Para el proyecto de la App, la actividad \textbf{1.3.1.2: Proyección de Ingresos y Flujo de Caja} es un entregable clave. Este flujo es crucial para el Análisis Financiero (A3) y la determinación de la viabilidad económica mediante los indicadores TIR y VPN.

El resumen del presupuesto que compone la línea base (identificado en el Acta de Constitución del Proyecto) es el siguiente:

\begin{table}[h!]
  \centering
  \begin{tabularx}{0.8\textwidth}{@{}Xr@{}}
    \toprule
    \textbf{Categoría de Costo (Año 1)} & \textbf{Presupuesto (COP)} \\ \midrule
    Inversión Inicial (Aporte Socios) & \$80,500,000 \\
    Costos Operativos (Nómina, Arriendo, Servicios) & \$229,000,000 \\
    Costos de Inversión (Equipos, Licencias) & \$75,500,000 \\ \midrule
    \textbf{PRESUPUESTO TOTAL ESTIMADO (AÑO 1)} & \textbf{\$309,500,000} \\ \bottomrule
  \end{tabularx}
  \caption{Resumen del Presupuesto (Año 1)}
\end{table}

\section{PLANIFICACIÓN DEL CONTROL Y SEGUIMIENTO DE LOS COSTOS DEL PROYECTO}

Para asegurar que los costos se mantengan dentro de la línea base aprobada (\$309.5M), se utiliza el proceso Controlar los Costos (7.4).

El control se realiza monitoreando el estado del proyecto y gestionando los cambios, utilizando herramientas clave como la \textbf{Gestión del Valor Ganado (EVM)}.

El EVM compara tres dimensiones:
\begin{itemize}
  \item \textbf{Valor Planificado (PV):} El presupuesto autorizado para el trabajo programado.
  \item \textbf{Costo Real (AC):} El costo real incurrido por el trabajo realizado.
  \item \textbf{Valor Ganado (EV):} El valor del trabajo completado.
\end{itemize}

El desempeño de costos se mide con el \textbf{Índice de Desempeño del Costo (CPI)}. Un CPI inferior a 1 indica que los gastos son mayores a los planificados para el trabajo realizado. Las desviaciones importantes pueden requerir el uso de Análisis de Reservas o la emisión de Solicitudes de Cambio formales para ajustar la Línea Base de Costos.

\end{document}
