\documentclass[11pt,letterpaper]{article}
\usepackage[utf8]{inputenc}
\usepackage[spanish]{babel}
\usepackage{geometry}
\usepackage{fancyhdr}
\usepackage{booktabs}
\usepackage{longtable}
\usepackage{array}
\usepackage{enumitem}
\usepackage{tikz}
\usepackage{hyperref}

\geometry{
  letterpaper,
  left=1in,
  right=1in,
  top=1in,
  bottom=1in
}

\pagestyle{fancy}
\fancyhf{} % Clear all header and footer fields
\fancyhead[L]{\textit{Alcance del Proyecto}}
\fancyfoot[C]{\thepage}
\renewcommand{\headrulewidth}{0.4pt}
\renewcommand{\footrulewidth}{0.4pt}

\usepackage{pgfplots}
\usepackage{pgfplotstable}
\pgfplotsset{compat=1.18}
\usetikzlibrary{backgrounds}

% Title and author formatting
\title{\textbf{Alcance del Proyecto de Aplicación Móvil de Gestión Energética}}
\author{
  Bryan Martinez Anzola \\
  Laura Yeraldin Herrera Martinez \\
  Julian Leonardo Avila Martinez \\[1em]
  \textit{F\'isica}\\
  \textit{Universidad Distrital Francisco José de Caldas}
}
\date{\today}

\newcolumntype{L}[1]{>{\raggedright\let\newline\\\arraybackslash\hspace{0pt}}m{#1}}

\begin{document}

\maketitle
\thispagestyle{empty}

\newpage

\section{Definición del Alcance}

El presente documento detalla el alcance para la elaboración de una
propuesta de diseño de una aplicación móvil \textbf{(qué)}. Este será
desarrollado por el equipo de proyecto \textbf{(quién)} y está dirigido a los
hogares de Bogotá con medidores inteligentes de Codensa S.A ESP
\textbf{(dónde)}. El proceso consiste en la formulación de una propuesta
integral, abarcando desde el análisis del problema hasta un estudio de
viabilidad financiera \textbf{(cómo)}, utilizando como insumo principal la
información de consumo energético de dichos medidores \textbf{(con qué)}.

A continuación, se desglosa el alcance del producto final y del proyecto que lo
enmarca.

A continuación, se presenta un resumen del alcance para el proyecto y para el
producto, seguido del desglose detallado de cada uno.

\paragraph{Resumen del Alcance del Proyecto.}
El proyecto se enfoca en la elaboración de una \textbf{propuesta de diseño}
completa para una aplicación móvil, concebida como un trabajo de grado. Este
proceso abarca desde la investigación inicial y el análisis del problema hasta
la entrega de un estudio de viabilidad financiera, junto con documentos clave
de gestión como la EAP y la Matriz de Marco Lógico. Queda explícitamente
fuera de este alcance el desarrollo de código o la implementación técnica de
la aplicación.

\paragraph{Resumen del Alcance del Producto.}
El producto consiste en el \textbf{documento de propuesta de diseño} para una
aplicación móvil dirigida a clientes de Codensa S.A ESP en Bogotá que
utilizan medidores inteligentes. La propuesta describirá funcionalidades
clave como el seguimiento del consumo en tiempo real, consejos de ahorro
energético y el cálculo del impacto ambiental. Asimismo, definirá el público
objetivo, un modelo de negocio de pago, su distribución vía Google Play Store
y los criterios de éxito para una eventual implementación.

\section{Alcance del Producto}

El producto no es la aplicación móvil funcional, sino la \textbf{propuesta
de diseño conceptual, técnico y financiero} para su posterior desarrollo.

\subsection{Descripción y Funcionalidades}
La aplicación tiene como objetivo principal gestionar la información de
consumo de energía en los hogares de Bogotá, procesando los datos de los
medidores inteligentes instalados por Codensa S.A ESP. Busca mitigar la
incertidumbre de los usuarios sobre su consumo y costo energético.

Las funcionalidades y beneficios contemplados en el diseño son:
\begin{itemize}
  \item Visualización del consumo diario para fomentar el ahorro.
  \item Provisión de consejos personalizados para el ahorro energético.
  \item Estimación del costo de la factura en tiempo real.
  \item Simulación de beneficios ante la implementación de energías renovables.
  \item Cálculo del impacto ambiental del consumo (emisiones de CO2).
  \item Comparación de datos históricos de consumo con meses anteriores.
\end{itemize}
Adicionalmente, la propuesta debe definir un protocolo de comunicación entre
la aplicación y el servidor del medidor, así como un proceso de análisis
matemático de los datos.

\subsection{Público Objetivo}
El producto está diseñado para los siguientes usuarios:
\begin{itemize}
  \item Clientes residenciales de Codensa S.A ESP en la ciudad de Bogotá.
  \item Específicamente, hogares que cuenten con medidores inteligentes
    de energía.
  \item El mercado potencial inicial se estima en 40,000 clientes, con una
    segmentación detallada por estrato socioeconómico.
\end{itemize}

\subsection{Aspectos Comerciales y Técnicos}
\begin{itemize}
  \item \textbf{Canal de Distribución:} La propuesta considerará Google
    Play Store como el canal de comercialización para dispositivos Android.
  \item \textbf{Modelo de Precios:} Se propone un modelo de aplicación paga,
    con un precio de referencia de \$1,000 pesos colombianos, basado en un
    análisis de mercado.
  \item \textbf{Estudio Técnico:} El diseño incluirá una definición de los
    requerimientos de software y hardware necesarios para el desarrollo.
\end{itemize}

\subsection{Criterios de Aceptación del Producto}
El éxito del producto diseñado, si fuese implementado, se mediría a través
del cumplimiento de los siguientes objetivos:
\begin{itemize}
  \item Minimizar la incertidumbre sobre el consumo y costo de la energía
    en los hogares objetivo.
  \item Contribuir a una disminución porcentual de las quejas sobre
    facturación en CODENSA.
  \item Fomentar un aumento porcentual en el pago oportuno de las facturas
    de energía por parte de los usuarios.
\end{itemize}

\section{Alcance del Proyecto}
El proyecto abarca todas las actividades necesarias para generar la
propuesta de diseño descrita anteriormente, en el marco de un proyecto de
grado.

\subsection{Entregables del Proyecto}
El entregable principal es el documento consolidado de la \textbf{Propuesta
de Diseño de la Aplicación Móvil}. Adicionalmente, se generarán los
siguientes entregables implícitos del trabajo de grado:
\begin{itemize}
  \item Análisis del problema y de objetivos.
  \item Selección y justificación de la estrategia óptima.
  \item Estructura Analítica del Proyecto (EAP).
  \item Matriz de Marco Lógico.
  \item Normograma aplicable al proyecto.
  \item Estudio financiero que determine la viabilidad del proyecto.
  \item Documento final con conclusiones y bibliografía.
\end{itemize}

\subsection{Exclusiones del Proyecto}
Para gestionar adecuadamente las expectativas de los interesados, se define
explícitamente lo que está fuera del alcance:
\begin{itemize}
  \item El desarrollo de software, la programación o la implementación
    física de la aplicación móvil.
  \item La instalación, operación o mantenimiento de los medidores
    inteligentes de energía, responsabilidad que recae en Codensa S.A ESP.
\end{itemize}

\subsection{Restricciones}
\begin{itemize}
  \item El proyecto debe concluir con un estudio que determine la
    viabilidad financiera de la implementación de la aplicación.
  \item La evaluación financiera debe considerar un período de vida útil
    de 5 años para la proyección del flujo de caja.
\end{itemize}

\subsection{Supuestos}
La planificación del proyecto se basa en los siguientes supuestos:
\begin{itemize}
  \item El proyecto es financieramente viable, según los indicadores de
    TIR y VPN obtenidos en el estudio económico.
  \item Las ganancias proyectadas se optimizan si la inversión inicial se
    asegura mediante un préstamo de una entidad financiera.
\end{itemize}


\end{document}
