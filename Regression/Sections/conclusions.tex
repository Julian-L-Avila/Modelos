Se identificó un patrón recurrente de disminución en las ventas durante el
segundo trimestre de cada año.
Este comportamiento es seguido por trimestres con mayores volúmenes de ventas,
siendo el cuarto trimestre el de mayor actividad, según los índices estacionales.

Mediante el uso del método de promedio móvil y la desestacionalización de los
datos históricos, se obtuvo una tendencia lineal que permitió proyectar las
ventas para el año siguiente.
La predicción estacionalizada conserva el comportamiento oscilante
característico de las ventas por trimestre.
