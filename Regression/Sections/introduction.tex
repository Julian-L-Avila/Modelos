Tener la capacidad de anticipar lo que ocurrirá en el futuro de una organización
o empresa es fundamental para su éxito a largo plazo.
En este contexto surge el concepto de \textbf{pronóstico}, entendido como una
estimación cuantitativa o cualitativa de uno o varios factores (variables) que
componen un evento futuro, basada en información actual o pasada.

Por ejemplo, las estimaciones del volumen de ventas trimestral de un producto
para el próximo año afectan directamente los programas de producción, los planes
de compra de materias primas, las políticas de inventarios y las metas de
ventas.
Un pronóstico erróneo puede ocasionar un aumento significativo en los costos
operativos de la empresa.

Para realizar un pronóstico, se suele comenzar con una revisión de los datos
históricos, lo cual permite entender mejor el patrón de ventas pasadas y, por
ende, generar predicciones más precisas sobre las ventas futuras.
Estos datos históricos conforman una \textbf{serie de tiempo}, es decir, un
conjunto de observaciones de una variable tomadas en puntos sucesivos en el
tiempo o a lo largo de períodos continuos \cite{Enders2015}.

Cuando el pronóstico se basa únicamente en los valores pasados de la variable en
cuestión, se utiliza lo que se conoce como \textbf{método de series de tiempo}.

Un proceso clave en el análisis de series de tiempo es la
\textbf{desestacionalización}, que consiste en eliminar el efecto estacional
(patrón de cambio que se repite con la misma frecuencia en cada período).
Una vez desestacionalizada la serie, las componentes periódicas se vuelven más
evidentes, lo que facilita la identificación de posibles tendencias \cite{Anderson}.
