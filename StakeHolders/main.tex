\documentclass[12pt,letterpaper]{article}
\usepackage[utf8]{inputenc}
\usepackage[spanish]{babel}
\usepackage{geometry}
\usepackage{fancyhdr}
\usepackage{booktabs}
\usepackage{longtable}
\usepackage{array}
\usepackage{enumitem}
\usepackage{tikz}
\usepackage{hyperref}

% Page geometry and layout
\geometry{
  letterpaper,
  left=1in,
  right=1in,
  top=1in,
  bottom=1in
}

% Professional header and footer
\pagestyle{fancy}
\fancyhf{} % Clear all header and footer fields
\fancyhead[L]{\textit{Análisis de Stakeholders: Gestión Energética Móvil}}
\fancyfoot[C]{\thepage}
\renewcommand{\headrulewidth}{0.4pt}
\renewcommand{\footrulewidth}{0.4pt}

\usepackage{pgfplots}
\usepackage{pgfplotstable}
\pgfplotsset{compat=1.18}
\usetikzlibrary{backgrounds}

\pgfplotstableread{
x y id
5.0 3.2 1
4.2 5.0 2
2.1 4.0 3
2.2 2.3 4
1.3 2.0 5
3.0 2.2 6
2.4 2.1 7
5.0 5.0 8
5.0 4.1 9
4.1 3.0 10
5.0 5.0 11
1.2 3.1 12
2.3 4.3 13
3.1 3.3 14
2.0 2.4 15
3.4 2.2 16
1.0 1.1 17
3.2 2.5 18
2.1 1.3 19
3.3 4.0 20
2.5 2.0 21
1.4 2.1 23
2.4 3.2 24
5.0 4.2 25
3.1 2.3 26
4.3 3.4 27
3.3 2.2 28
3.0 3.0 29
2.2 1.2 30
}\StakeholdersFavor

\pgfplotstableread{
x y id
4.0 2.4 22
4.1 4.4 31
4.4 3.3 32
3.2 3.1 33
}\StakeholdersContra

\pgfplotsset{
    powerinterestmatrix/.style={
        width=12cm,
        height=12cm,
        xlabel={Interés $\rightarrow$},
        ylabel={Poder $\rightarrow$},
        xlabel style={at=(current axis.right of origin), anchor=west},
        ylabel style={at=(current axis.above origin), anchor=south, rotate=-90},
        xmin=0, xmax=6,
        ymin=0, ymax=6,
        xtick={1,2,3,4,5},
        ytick={1,2,3,4,5},
        axis lines=middle,
        tick style={color=black},
        grid=major,
        grid style={dashed, gray!40},
        enlarge x limits={abs=0.5},
        enlarge y limits={abs=0.5},
    }
}

\newcommand{\PowerInterestMatrix}[4]{
\begin{figure}[htbp!]
    \centering
    \begin{tikzpicture}
        \begin{axis}[
            powerinterestmatrix, % Aplicar el estilo base
            title={#3} % Usar el caption como título del gráfico
        ]
            % --- Capa de Fondo para los Cuadrantes ---
            \begin{scope}[on background layer]
                \fill[blue!10] (axis cs:3,3) rectangle (axis cs:6,6);
                \fill[green!10] (axis cs:0,3) rectangle (axis cs:3,6);
                \fill[orange!10] (axis cs:3,0) rectangle (axis cs:6,3);
                \fill[red!10] (axis cs:0,0) rectangle (axis cs:3,3);
            \end{scope}
            % --- Etiquetas de los Cuadrantes ---
            \node[align=center, font=\bfseries] at (axis cs:4.5, 4.5) {Gestionar\\de Cerca};
            \node[align=center] at (axis cs:1.5, 4.5) {Mantener\\Satisfecho};
            \node[align=center] at (axis cs:4.5, 1.5) {Mantener\\Informado};
            \node[align=center] at (axis cs:1.5, 1.5) {Monitorear};
            % --- Gráfico de Dispersión (Scatter Plot) ---
            \addplot+[
                only marks,
                % La metadata (el ID) se usa para las etiquetas de los nodos
                point meta=explicit symbolic,
                % Extrae el valor de la columna 'id' y lo asigna a \labelval
                visualization depends on={\thisrow{id} \as \labelval},
                % Muestra la etiqueta \labelval cerca de cada coordenada
                nodes near coords={%
                    \pgfmathtruncatemacro{\myint}{\labelval}% Convertir a entero
                    \myint% Mostrar el entero
                },
                nodes near coords style={
                    font=\tiny,
                    anchor=south west,
                },
                #2 % Aplica opciones específicas (color, forma del marcador)
            ] table [x=x, y=y, meta=id] {#1};
        \end{axis}
    \end{tikzpicture}
    \caption{#3}
    \label{#4}
\end{figure}
}

% Title and author formatting
\title{\textbf{Análisis de Stakeholders para el Proyecto de Aplicación Móvil de Gestión Energética}}
\author{
  Bryan Martinez Anzola \\
  Laura Yeraldin Herrera Martinez \\
  Julian Leonardo Avila Martinez \\[1em]
  \textit{Facultad de Ingeniería}\\
  \textit{Universidad Distrital Francisco José de Caldas}
}
\date{\today}

\newcolumntype{L}[1]{>{\raggedright\let\newline\\\arraybackslash\hspace{0pt}}m{#1}}

\begin{document}

\maketitle
\thispagestyle{empty}

\newpage

\section{Identificación y Justificación de Stakeholders}

A continuación, se presenta una tabla detallada con la identificación de los stakeholders clave del proyecto, clasificados por categoría, junto con la justificación de su interés en el mismo.

\begin{longtable}{p{0.5cm} L{4.5cm} L{9.5cm}}
  \toprule
  \textbf{ID} & \textbf{Stakeholder} & \textbf{Justificación del Interés} \\
  \midrule
  \endhead
  \bottomrule
  \endfoot

  \multicolumn{3}{l}{\textbf{Categoría: Directamente Mencionados en el Proyecto}} \\
  \midrule
  1 & Consumidores de Energía Eléctrica & Usuarios finales de la aplicación, cuyo problema de incertidumbre sobre el consumo energético se busca resolver.  \\
  2 & Codensa S.A ESP & Proveedor de energía, instalador de medidores inteligentes y cliente potencial interesado en reducir quejas de usuarios.  \\
  3 & Comisión de Regulación de Energía y Gas (CREG) & Entidad supervisora de servicios públicos con interés en herramientas que afectan la relación empresa-consumidor.  \\
  4 & Empresas Públicas de energía eléctrica del país & Identificadas como posibles clientes y un mercado de expansión para el proyecto.  \\
  5 & Proveedores de Internet & Stakeholder externo cuyo servicio es indispensable para el funcionamiento de la aplicación.  \\
  6 & Clientes de publicidad & Representan una fuente de ingresos al estar interesados en mostrar su publicidad en la aplicación.  \\
  7 & Universidad Distrital Francisco José de Caldas & Institución académica que respalda y en cuyo marco se desarrolla el proyecto.  \\
  8 & Gerente de proyecto & Responsable de la gestión, planificación y ejecución exitosa del proyecto.  \\
  9 & Ingeniero desarrollador & Recurso humano técnico encargado de la creación, implementación y mantenimiento del software.  \\
  10 & Técnicos operativos & Responsables de la validación y verificación del sistema para asegurar el cumplimiento de requerimientos.  \\
  11 & Socios del proyecto & Aportan la inversión inicial y esperan un retorno financiero y estratégico.  \\
  12 & Google & Propietario de Google Play Store, el canal de distribución exclusivo para la aplicación.  \\
  \midrule
  \multicolumn{3}{l}{\textbf{Categoría: Gobierno y Reguladores}} \\
  \midrule
  13 & Superintendencia de Servicios Públicos Domiciliarios & Entidad nacional de vigilancia interesada en la protección de los derechos del consumidor.  \\
  14 & Ministerio de Minas y Energía & Define la política energética del país; interés en tecnologías que promuevan la eficiencia energética.  \\
  15 & Alcaldía Mayor de Bogotá & Interesada en iniciativas que mejoren la calidad de los servicios públicos para los ciudadanos.  \\
  16 & Secretaría Distrital de Ambiente de Bogotá & Interesada en el impacto ambiental positivo del proyecto al incentivar la reducción del consumo energético.  \\
  17 & Departamento Administrativo Nacional de Estadística (DANE) & Stakeholder indirecto cuya información socioeconómica es fundamental para el estudio de mercado.  \\
  \midrule
  \multicolumn{3}{l}{\textbf{Categoría: Corporativos y Comerciales}} \\
  \midrule
  18 & Fabricantes de medidores inteligentes & Su tecnología es la fuente de datos; el éxito de la app puede impulsar la demanda de sus productos.  \\
  19 & Otras empresas de servicios públicos & Mercado potencial que observa el desempeño de esta iniciativa como un modelo a seguir.  \\
  20 & Bancos y Entidades Financieras & Fuente de financiación alternativa, con interés en la viabilidad y rentabilidad del proyecto.  \\
  21 & Agencias de Publicidad y Marketing & Intermediarios para conseguir clientes de publicidad, una fuente de ingreso clave.  \\
  22 & Desarrolladores de aplicaciones competidoras & Su existencia valida el mercado, pero representan una competencia directa.  \\
  23 & Fabricantes de teléfonos inteligentes & La aplicación debe ser compatible con sus dispositivos para llegar a los usuarios.  \\
  24 & Empresas de Ciberseguridad & La aplicación manejará datos sensibles de consumo, por lo que la seguridad es un factor crítico.  \\
  \midrule
  \multicolumn{3}{l}{\textbf{Categoría: Usuarios y Comunidad}} \\
  \midrule
  25 & Hogares de estratos socioeconómicos 2 en adelante & Segmento de mercado objetivo que posee smartphones y capacidad de pago.  \\
  26 & Juntas de Acción Comunal & Canales de difusión y validación del proyecto a nivel comunitario.  \\
  27 & Asociaciones de Consumidores & Defienden los derechos de los usuarios y valoran herramientas de transparencia y control.  \\
  28 & Administradores de Propiedad Horizontal & Podrían promover el uso de la app para fomentar el ahorro energético comunitario.  \\
  29 & Medios de Comunicación & Tienen interés en noticias sobre innovación y pueden influir en la percepción pública.  \\
  30 & Instituciones Educativas y Académicas & Interesadas en el proyecto como caso de estudio sobre tecnología y emprendimiento.  \\
  \midrule
  \multicolumn{3}{l}{\textbf{Categoría: Potenciales Opositores y Grupos Críticos}} \\
  \midrule
  31 & Grupos de Defensa del Consumidor y Privacidad de Datos & Preocupados por la recolección de datos que puedan revelar hábitos personales y constituir una invasión de la privacidad.  \\
  32 & Organizaciones de Ciberseguridad y Derechos Digitales & Centrados en vulnerabilidades técnicas que podrían comprometer la seguridad de los datos y del suministro energético.  \\
  33 & Defensores de la Equidad Social y la Inclusión & Argumentarían que el proyecto podría ampliar la brecha digital al excluir a hogares de bajos ingresos.  \\
\end{longtable}

\newpage
\section{Análisis de Posición, Poder e Interés}

La siguiente tabla clasifica a los stakeholders según su posición frente al proyecto (a favor o en contra) y cuantifica su nivel de poder e interés en una escala de 1.0 a 5.0.

\begin{longtable}{l L{5.5cm} c c c}
  \toprule
  \textbf{ID} & \textbf{Stakeholder} & \textbf{Posición} & \textbf{Poder} & \textbf{Interés} \\
  \midrule
  \endhead
  \bottomrule
  \endfoot

  \multicolumn{5}{l}{\textbf{Directamente Mencionados}} \\
  \midrule
  1 & Consumidores de Energía Eléctrica & A favor & 3.2 & 5.0 \\
  2 & Codensa S.A ESP & A favor & 5.0 & 4.2 \\
  3 & Comisión de Regulación de Energía y Gas (CREG) & A favor & 4.0 & 2.1 \\
  4 & Empresas Públicas de energía eléctrica del país & A favor & 2.3 & 2.2 \\
  5 & Proveedores de Internet & A favor & 2.0 & 1.3 \\
  6 & Clientes de publicidad & A favor & 2.2 & 3.0 \\
  7 & Universidad Distrital Francisco José de Caldas & A favor & 2.1 & 2.4 \\
  8 & Gerente de proyecto & A favor & 5.0 & 5.0 \\
  9 & Ingeniero desarrollador & A favor & 4.1 & 5.0 \\
  10 & Técnicos operativos & A favor & 3.0 & 4.1 \\
  11 & Socios del proyecto & A favor & 5.0 & 5.0 \\
  12 & Google & A favor & 3.1 & 1.2 \\
  \midrule
  \multicolumn{5}{l}{\textbf{Gobierno y Reguladores}} \\
  \midrule
  13 & Superintendencia de Servicios Públicos Domiciliarios & A favor & 4.3 & 2.3 \\
  14 & Ministerio de Minas y Energía & A favor & 3.3 & 3.1 \\
  15 & Alcaldía Mayor de Bogotá & A favor & 2.4 & 2.0 \\
  16 & Secretaría Distrital de Ambiente de Bogotá & A favor & 2.2 & 3.4 \\
  17 & DANE & A favor & 1.1 & 1.0 \\
  \midrule
  \multicolumn{5}{l}{\textbf{Corporativos y Comerciales}} \\
  \midrule
  18 & Fabricantes de medidores inteligentes & A favor & 2.5 & 3.2 \\
  19 & Otras empresas de servicios públicos & A favor & 1.3 & 2.1 \\
  20 & Bancos y Entidades Financieras & A favor & 4.0 & 3.3 \\
  21 & Agencias de Publicidad y Marketing & A favor & 2.0 & 2.5 \\
  22 & Desarrolladores de aplicaciones competidoras & En contra & 2.4 & 4.0 \\
  23 & Fabricantes de teléfonos inteligentes & A favor & 2.1 & 1.4 \\
  24 & Empresas de Ciberseguridad & A favor & 3.2 & 2.4 \\
  \midrule
  \multicolumn{5}{l}{\textbf{Usuarios y Comunidad}} \\
  \midrule
  25 & Hogares de estratos 2 en adelante & A favor & 4.2 & 5.0 \\
  26 & Juntas de Acción Comunal & A favor & 2.3 & 3.1 \\
  27 & Asociaciones de Consumidores & A favor & 3.4 & 4.3 \\
  28 & Administradores de Propiedad Horizontal & A favor & 2.2 & 3.3 \\
  29 & Medios de Comunicación & A favor & 3.0 & 3.0 \\
  30 & Instituciones Educativas y Académicas & A favor & 1.2 & 2.2 \\
  \midrule
  \multicolumn{5}{l}{\textbf{Potenciales Opositores y Grupos Críticos}} \\
  \midrule
  31 & Grupos de Defensa del Consumidor y Privacidad & En contra & 4.4 & 4.1 \\
  32 & Organizaciones de Ciberseguridad y Derechos Digitales & En contra & 3.3 & 4.4 \\
  33 & Defensores de la Equidad Social y la Inclusión & En contra & 3.1 & 3.2 \\
\end{longtable}

\newpage
\section{Visualización Gráfica: Matriz Poder vs. Interés}

\subsection{Stakeholders a Favor del Proyecto}

La siguiente matriz posiciona a los stakeholders que apoyan el proyecto. Aquellos en el cuadrante superior derecho (ej. Socios, Gerente de proyecto) son actores clave que deben ser gestionados de cerca.

\PowerInterestMatrix
    {\StakeholdersFavor}
    {mark=*, blue}
    {Matriz de Poder vs. Interés para stakeholders a favor del proyecto.}
    {fig:a_favor_pgf}

\newpage
\subsection{Stakeholders en Contra del Proyecto}

Esta matriz identifica a los stakeholders opositores. El grupo de ``Defensa del
Consumidor y Privacidad'' (31) se ubica como un actor de alto poder y alto
interés, requiriendo una gestión proactiva y estratégica.

\PowerInterestMatrix
    {\StakeholdersContra}
    {mark=triangle*, red, mark size=3pt}
    {Matriz de Poder vs. Interés para stakeholders en contra del proyecto.}
    {fig:en_contra_pgf}

\section{Estrategias para la Gestión de Stakeholders Opositores}

Para asegurar la viabilidad del proyecto, es crucial gestionar de manera proactiva a los stakeholders que puedan presentar oposición.  A continuación, se detallan las estrategias para mitigar sus preocupaciones y reducir su poder de influencia negativa.

\subsection{Grupos de Defensa del Consumidor y Privacidad de Datos}
\begin{description}
  \item[Preocupación Principal:] Recolección masiva de datos personales que revelan hábitos de vida y privacidad del usuario.
  \item[Estrategias de Mitigación:]
    \begin{itemize}[leftmargin=*]
      \item \textbf{Transparencia Radical:} Desarrollar políticas de privacidad y términos de servicio en lenguaje claro y accesible. Explicar de manera explícita qué datos se recolectan, su finalidad, cómo se almacenan y si se comparten con terceros.
      \item \textbf{Consentimiento Granular:} Implementar un sistema donde los usuarios puedan elegir qué tipo de datos desean compartir, otorgándoles un mayor control sobre su información.
      \item \textbf{Anonimización de Datos:} Garantizar que todos los datos para análisis agregados sean completamente anonimizados y disociados de la identidad del usuario.
      \item \textbf{Alianza Proactiva:} Invitar a representantes de estas organizaciones a mesas de trabajo y auditorías de privacidad previas al lanzamiento para convertirlos en validadores del proyecto.
    \end{itemize}
\end{description}

\subsection{Organizaciones de Ciberseguridad y Derechos Digitales}
\begin{description}
  \item[Preocupación Principal:] Riesgo de vulnerabilidades en la aplicación, medidores e infraestructura que podrían derivar en brechas de datos o disrupciones del servicio.
  \item[Estrategias de Mitigación:]
    \begin{itemize}[leftmargin=*]
      \item \textbf{Auditorías de Seguridad Externas:} Contratar firmas de ciberseguridad para realizar pruebas de penetración y auditorías de código. Publicar un resumen de los resultados para generar confianza.
      \item \textbf{Arquitectura de Seguridad Robusta:} Implementar cifrado de extremo a extremo y utilizar las mejores prácticas de seguridad en el desarrollo (ej., OWASP Top 10).
      \item \textbf{Programa de Recompensas por Fallos (Bug Bounty):} Crear un programa que incentive a hackers éticos a encontrar y reportar vulnerabilidades a cambio de una recompensa.
      \item \textbf{Comunicación Técnica Detallada:} Elaborar un ``whitepaper''
          técnico que detalle las medidas de seguridad implementadas para que
          los expertos puedan evaluar la robustez del sistema.
    \end{itemize}
\end{description}

\subsection{Defensores de la Equidad Social y la Inclusión}
\begin{description}
  \item[Preocupación Principal:] El proyecto podría excluir a la población más vulnerable (estrato 1) y fomentar prácticas discriminatorias.
  \item[Estrategias de Mitigación:]
    \begin{itemize}[leftmargin=*]
      \item \textbf{Creación de una Versión Inclusiva:} Desarrollar una versión de la aplicación que funcione con tecnología USSD o SMS, sin requerir un smartphone o conexión de datos permanente.
      \item \textbf{Alianzas con Programas Gubernamentales:} Colaborar con entidades públicas para integrar la aplicación en programas de subsidio o de inclusión digital.
      \item \textbf{Política de No Discriminación por Datos:} Establecer y comunicar públicamente una política estricta y auditable que prohíba el uso de datos de consumo para fines adversos al usuario.
      \item \textbf{Proyectos Piloto Comunitarios:} Trabajar con Juntas de Acción Comunal para instalar puntos de acceso Wi-Fi y ofrecer capacitación sobre el uso y beneficios de la aplicación.
    \end{itemize}
\end{description}

\end{document}
