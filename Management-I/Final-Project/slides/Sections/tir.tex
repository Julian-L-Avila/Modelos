\begin{frame}
  \frametitle{Análisis Financiero: Proyección a 5 Años}
  \framesubtitle{Cifras en COP}

  \resizebox{\textwidth}{!}{%
    \begin{tabular}{lS[table-format=8.0] S[table-format=8.0] S[table-format=9.0] S[table-format=9.0] S[table-format=9.0]}
      \toprule
      \textbf{Concepto} & {\textbf{Año 1}} & {\textbf{Año 2}} & {\textbf{Año 3}} & {\textbf{Año 4}} & {\textbf{Año 5}} \\
      \midrule
      Ingresos Totales & 3428571 & 6857142 & 10285713 & 10285713 & 10285713 \\
      (-) Costos Directos & 2400000 & 4800000 & 7200000 & 7200000 & 7200000 \\
      \midrule
      \textbf{Utilidad Bruta} & \textbf{1028571} & \textbf{2057142} & \textbf{3085713} & \textbf{3085713} & \textbf{3085713} \\
      \midrule
      (-) Gastos Operativos & 570000 & 1140000 & 1710000 & 1710000 & 1710000 \\
      \midrule
      \textbf{Utilidad Op. (UAII)} & \textbf{458571} & \textbf{917142} & \textbf{1375713} & \textbf{1375713} & \textbf{1375713} \\
      \bottomrule
    \end{tabular}%
  }
\end{frame}

\begin{frame}
  \frametitle{Indicador Clave: Tasa Interna de Retorno (TIR)}

  \begin{block}{Definición}
    Representa la rentabilidad porcentual promedio anual que genera el
    proyecto sobre la inversión inicial.
  \end{block}

  \begin{columns}[T]
    \begin{column}{0.5\textwidth}
      \begin{block}{Flujo de Caja Proyectado}
        \begin{itemize}
          \item \textbf{Año 0:} \num{-2400000}
          \item \textbf{Año 1:} \num{458571}
          \item \textbf{Año 2:} \num{917142}
          \item \textbf{Año 3:} \num{1375713}
          \item \textbf{Año 4:} \num{1375713}
          \item \textbf{Año 5:} \num{1375713}
        \end{itemize}
      \end{block}
    \end{column}
    \begin{column}{0.5\textwidth}
      \begin{alertblock}{Resultado}
        \centering
        \Huge{\textbf{29.1\%}}
      \end{alertblock}
    \end{column}
  \end{columns}
\end{frame}

\begin{frame}
  \frametitle{Conclusiones de Viabilidad Financiera}

  \begin{itemize}
    \item<1-> \textbf{Alta Rentabilidad:} Una TIR del \textbf{29.1\%} es
      superior a las tasas de oportunidad del mercado (15-20\%), lo que
      indica que el proyecto es una inversión muy atractiva.
      \vspace{2em}

    \item<2-> \textbf{Punto de Equilibrio:} El flujo de caja acumulado se
      torna positivo durante el \textbf{tercer año}. La inversión inicial se
      recupera en su totalidad en este punto.
      \vspace{2em}

    \item<3-> \textbf{Validación del Modelo:} El hito financiero del Año 3
      coincide con el último escalamiento a 9 colegios, validando que este
      tamaño es sostenible y consolida la viabilidad a largo plazo.
  \end{itemize}
\end{frame}
